\section{Modelle}

\begin{frame}[c]
  \visible<\theFirstElement->{
    \begin{itemize}
    \item Problem: \\
      Gängige Grundflächenmodelle erscheinen unzureichend
      
    \item Lösungsansatz: \\
      GAM (Generalized Additive Model) und \\
      GAMLSS (Generalized Additive Model for Location, Scale, and Shape)
    \end{itemize}}
\end{frame}

\subsection{Hintergrund}

\begin{frame}[c]
  \visible<\theFirstElement->{
    {\scriptsize
      \mycaption{Tabelle 1}{Unterschiede zwischen linearem Modell (LM), generalisiertem linearen Model (GLM), GAM und GAMLSS.}
      \begin{tabular}{l l l l}
      \toprule
      Modelltyp & modellierter Zusammenhang & unterstellte Verteilung & geschätzter Parameter \\
      \midrule
      LM & linear & Normal & Mittelwert \\
      GLM & linear & Exponentialfamilie & Mittelwert \\
      GAM & beliebig & Exponentialfamilie & Mittelwert \\
      GAMLSS & beliebig & beliebig & beliebig \\
      \bottomrule
    \end{tabular}}
  }
\end{frame}

\subsection{Ergebnisse}

\begin{frame}[c]
  \visible<\theFirstElement->{
    \centerline{
      \begin{minipage}{0.9\textwidth}
        \includegraphics[width=1.0\textwidth]{../../Graphics/Presentation/AgeBasalAreaObservationsGAM2PredictionsBeech.pdf}
        \captionsep{}
        \mycaption{Abbildung 3}{Beobachtete (Punkte) und durch GAM vorhergesagte (farbige Linien) Grundfläche \(G\) über Alter bei Buche. Farben deuten die jeweilige EKL an.}
      \end{minipage}}}
\end{frame}

\begin{frame}[c]
  \visible<\theFirstElement->{
    \centerline{
      \begin{minipage}{0.9\textwidth}
        \includegraphics[width=1.0\textwidth]{../../Graphics/Presentation/TopHeightBasalAreaObservationsGAM2PredictionsBeech.pdf}
        \captionsep{}
        \mycaption{Abbildung 4}{Beobachtete (Punkte) und durch GAM vorhergesagte (farbige Linien) Grundfläche \(G\) über Oberhöhe \(h_{100}\) bei Buche. Farben deuten die jeweilige EKL an.}
      \end{minipage}}}
\end{frame}

\begin{frame}[c]
  \visible<\theFirstElement->{
    \centerline{
      \begin{minipage}{0.9\textwidth}
        \includegraphics[width=1.0\textwidth]{../../Graphics/Presentation/AgeBasalAreaObservationsGAMLSS3PredictionsBeech.pdf}
        \captionsep{}
        \mycaption{Abbildung 5}{Beobachtete (Punkte) und durch GAMLSS vorhergesagte (farbige Linien) Grundfläche \(G\) über Alter bei Buche. Farben deuten die jeweilige EKL an.}
      \end{minipage}}}
\end{frame}

\begin{frame}[c]
  \visible<\theFirstElement->{
    \centerline{
      \begin{minipage}{0.9\textwidth}
        \includegraphics[width=1.0\textwidth]{../../Graphics/Presentation/TopHeightBasalAreaObservationsGAMLSS3PredictionsBeech.pdf}
        \captionsep{}
        \mycaption{Abbildung 6}{Beobachtete (Punkte) und durch GAMLSS vorhergesagte (farbige Linien) Grundfläche \(G\) über Oberhöhe \(h_{100}\) bei Buche. Farben deuten die jeweilige EKL an.}
      \end{minipage}}}
\end{frame}

%%% Local Variables:
%%% mode: latex
%%% TeX-master: "MasArPresentation.tex"
%%% End:
