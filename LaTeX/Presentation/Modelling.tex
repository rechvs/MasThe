\section{Modelle}

\subsection{Bisherige Modelle}
\begin{frame}[c]
  \visible<\theFirstElement->{
    \begin{itemize}
    \item<\theFirstElement-> Sterba ([??]): erscheint grds. ungeeignet
    \item<\theSecondElement-> Wördehoff ([??]): besser, aber bonitätsunempfindlich
    \end{itemize}}

  \visible<\theSecondElement->{
    \begin{center}
      \begin{minipage}[t]{0.975\linewidth}
        {\scriptsize
        \mycaption{Tabelle 1}{Vergleich der Grundfläche (\(G\)) bei gegebener Oberhöhe (\(h_{100}\)) in verschiedenen Ertragsklassen für Buche, mäßige Durchforstung (Quelle: [??]).}
        \begin{tabular}{l c c c c}
          \toprule
          & \multicolumn{4}{c}{\(G\) [\si{\square\meter\per\hectare}]} \\ \cline{2-5}
          \(h_{100}\) [\si{\meter}] & I. EKL & II. EKL & III. EKL & IV. EKL \\
          \midrule
          15 & \(18.25\) & \(18.9 < x < 20.57\) & \(20.95 < x < 22.35\) & \(22.78 < x < 23.96\) \\
          25 & \(25.6 < x < 26.8\) & \(27.76 < x < 28.58\) & \(29.3 < x < 29.7\) & \(30.72 < x < 30.97\) \\
          \bottomrule
        \end{tabular}}
      \end{minipage}
    \end{center}}
\end{frame}

\subsection{GAM und GAMLSS \textendash{} Hintergrund}

\begin{frame}[c]
  \visible<\theFirstElement->{
    \begin{center}
      \begin{minipage}[t]{0.95\linewidth}
        {\scriptsize
          \mycaption{Tabelle 1}{Unterschiede zwischen linearem Modell (LM), generalisiertem linearen Model (GLM), GAM und GAMLSS.}
          \begin{tabular}{l l l l}
            \toprule
            Modelltyp & modellierter Zusammenhang & unterstellte Verteilung & geschätzter Parameter \\
            \midrule
            LM & linear & Normal & Mittelwert \\
            GLM & linear & Exponentialfamilie & Mittelwert \\
            GAM & beliebig & Exponentialfamilie & Mittelwert \\
            GAMLSS & beliebig & beliebig & beliebig \\
            \bottomrule
          \end{tabular}}
        \end{minipage}
      \end{center}}
\end{frame}

\subsection{Ergebnisse}

\begin{frame}[c]
  \visible<\theFirstElement->{
    \centerline{
      \begin{minipage}{0.9\textwidth}
        \includegraphics[width=1.0\textwidth]{../../Graphics/Presentation/AgeBasalAreaObservationsGAM2PredictionsBeech.pdf}
        \captionsep{}
        \mycaption{Abbildung 3}{Beobachtete (Punkte) und durch GAM vorhergesagte (farbige Linien) Grundfläche \(G\) über Alter bei Buche. Farben deuten die jeweilige EKL an.}
      \end{minipage}}}
\end{frame}

\begin{frame}[c]
  \visible<\theFirstElement->{
    \centerline{
      \begin{minipage}{0.9\textwidth}
        \includegraphics[width=1.0\textwidth]{../../Graphics/Presentation/TopHeightBasalAreaObservationsGAM2PredictionsBeech.pdf}
        \captionsep{}
        \mycaption{Abbildung 4}{Beobachtete (Punkte) und durch GAM vorhergesagte (farbige Linien) Grundfläche \(G\) über Oberhöhe \(h_{100}\) bei Buche. Farben deuten die jeweilige EKL an.}
      \end{minipage}}}
\end{frame}

\begin{frame}[c]
  \visible<\theFirstElement->{
    \centerline{
      \begin{minipage}{0.9\textwidth}
        \includegraphics[width=1.0\textwidth]{../../Graphics/Presentation/AgeBasalAreaObservationsGAMLSS3PredictionsBeech.pdf}
        \captionsep{}
        \mycaption{Abbildung 5}{Beobachtete (Punkte) und durch GAMLSS vorhergesagte (farbige Linien) Grundfläche \(G\) über Alter bei Buche. Farben deuten die jeweilige EKL an.}
      \end{minipage}}}
\end{frame}

\begin{frame}[c]
  \visible<\theFirstElement->{
    \centerline{
      \begin{minipage}{0.9\textwidth}
        \includegraphics[width=1.0\textwidth]{../../Graphics/Presentation/TopHeightBasalAreaObservationsGAMLSS3PredictionsBeech.pdf}
        \captionsep{}
        \mycaption{Abbildung 6}{Beobachtete (Punkte) und durch GAMLSS vorhergesagte (farbige Linien) Grundfläche \(G\) über Oberhöhe \(h_{100}\) bei Buche. Farben deuten die jeweilige EKL an.}
      \end{minipage}}}
\end{frame}

%%% Local Variables:
%%% mode: latex
%%% TeX-master: "MasArPresentation.tex"
%%% End:
