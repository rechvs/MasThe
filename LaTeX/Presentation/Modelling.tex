\section{Modelle}

\subsection{Bisherige Modelle}
\begin{frame}[c]
  \visible<\theFirstElement->{
    \begin{itemize}
    \item<\theFirstElement-> Sterba (\Sterba{}): erscheint grds. nicht geeignet
    \item<\theSecondElement-> Wördehoff (\Woerdehoff{}): besser, aber noch bonitätsunempfindlich und deswegen nicht geeignet für den Ansatz von \DoebbelerSpellmann{} und \SpellmannEtAl{}
    \end{itemize}}

  \visible<\theSecondElement->{
    \begin{center}
      \begin{minipage}[t]{0.975\linewidth}
        {\scriptsize
          \mycaption{Tabelle 1}{Vergleich der Grundfläche (\(G\)) bei gegebener Oberhöhe (\(h_{100}\)) in verschiedenen Ertragsklassen für Buche, mäßige Durchforstung (Quelle: \Schober{}).}
          \begin{tabular}{l c c c c}
            \toprule
            & \multicolumn{4}{c}{\(G\) [\si{\square\meter\per\hectare}]} \\ \cline{2-5}
            \(h_{100}\) [\si{\meter}] & I. EKL & II. EKL & III. EKL & IV. EKL \\
            \midrule
            15 & \(18.25\) & \(18.9 < x < 20.57\) & \(20.95 < x < 22.35\) & \(22.78 < x < 23.96\) \\
            25 & \(25.6 < x < 26.8\) & \(27.76 < x < 28.58\) & \(29.3 < x < 29.7\) & \(30.72 < x < 30.97\) \\
            \bottomrule
          \end{tabular}}
      \end{minipage}
    \end{center}}
\end{frame}

\subsection{GAM und GAMLSS  \textendash{} Hintergrund}

\begin{frame}[c]
  \visible<\theFirstElement->{
    \begin{center}
      \begin{minipage}[t]{0.975\linewidth}
        {\scriptsize
          \mycaption{Tabelle 1}{Unterschiede zwischen linearen Modellen (LM), generalisierten linearen Modellen (GLM), GAM und GAMLSS.}
          \begin{tabular}{l l l l}
            \toprule
            Modelltyp & modellierter Zusammenhang & unterstellte Verteilung & geschätzter Parameter \\
            \midrule
            LM & linear & Normal & Mittelwert \\
            GLM & linear & Exponentialfamilie & Mittelwert \\
            GAM & beliebig & Exponentialfamilie & Mittelwert \\
            GAMLSS & beliebig & beliebig & beliebig \\
            \bottomrule
          \end{tabular}}
      \end{minipage}
    \end{center}}

  \visible<\theSecondElement->{
    \begin{center}
      \begin{minipage}[t]{0.825\linewidth}
        {\scriptsize
          \mycaption{Tabelle 2}{In \texttt{R} eingesetzte Gleichungen und Verteilungen der hier vorgestellten unbeschränkten GAMLSS. \texttt{h100.EKL.I} ist die Entwicklungsstufenvariable. \texttt{SI.h100.diff.EKL.I} ist die Bontitätsvariable.}
          \begin{tabular}{l l l}
            \toprule
            Baumart & Gleichung & Verteilung \\
            \midrule
            Buche & \texttt{gha \textasciitilde{} ps(h100.EKL.I) + ps(SI.h100.diff.EKL.I)} & \texttt{BCCGo()} \\
            Fichte & \texttt{gha \textasciitilde{} ps(h100.EKL.I) + ps(SI.h100.diff.EKL.I)} & \texttt{BCCGo()} \\
            \bottomrule
          \end{tabular}}
      \end{minipage}
    \end{center}}
  
  \visible<\theSecondElement->{
    \begin{center}
      \begin{minipage}[t]{0.925\linewidth}
        {\scriptsize
          \mycaption{Tabelle 3}{In \texttt{R} eingesetzte Gleichungen und Verteilungen der hier vorgestellten beschränkten GAMLSS. \texttt{h100.EKL.I} ist die Entwicklungsstufenvariable. \texttt{SI.h100.diff.EKL.I} ist die Bontitätsvariable.}
          \begin{tabular}{l l l}
            \toprule
            Baumart & Gleichung & Verteilung \\
            \midrule
            Buche & \texttt{gha \textasciitilde{} pbm(h100.EKL.I) + SI.h100.diff.EKL.I} & \texttt{BCCGo()} \\
            Fichte & \texttt{gha \textasciitilde{} lhs(x = h100.EKL.I, c = 22) + SI.h100.diff.EKL.I} & \texttt{BCCGo()} \\
            \bottomrule
          \end{tabular}}
      \end{minipage}
    \end{center}}
\end{frame}

\subsection{Unbeschränkte GAMLSS \textendash{} Ergebnisse}

\begin{frame}[c]
  \only<\theFirstElement>{
    \centerline{
      \begin{minipage}{0.9\textwidth}
        \includegraphics[width=1.0\textwidth]{../../Graphics/Presentation/TermEffectsGAMLSSUnconstrained.pdf}
        \captionsep{}
        \mycaption{Abbildung 3}{Effekt der Entwicklungsstufenvariable (\texttt{h100.EKL.I}, oben) und der Bonitätsvariable (\texttt{SI.h100.diff.EKL.I}, unten) im unbeschränkten GAMLSS für Buche (links) und Fichte (rechts).  Gestrichelte orange Linien markieren den Standardfehler.  Senkrechte graue Striche repräsentieren beobachtete Variablenwerte.}
      \end{minipage}}}

  \only<\theSecondElement>{
    \centerline{
      \begin{minipage}{0.9\textwidth}
        \includegraphics[width=1.0\textwidth]{../../Graphics/Presentation/StandAgeBasalAreaObservationsGAMLSSUnconstrainedPredictions.pdf}
        \captionsep{}
        \mycaption{Abbildung 4}{Beobachtete (Punkte) und durch unbeschränkte GAMLSS vorhergesagte (farbige Linien) Grundfläche (\(G\)) über Alter bei Buche (links) und Fichte (rechts).  Schwarze Linien verbinden Beobachtungen einer Parzelle.  Farben deuten die jeweilige Ertragsklasse an.}
      \end{minipage}}}

  \only<\theThirdElement>{
    \centerline{
      \begin{minipage}{0.9\textwidth}
        \includegraphics[width=1.0\textwidth]{../../Graphics/Presentation/TopHeightBasalAreaObservationsGAMLSSUnconstrainedPredictions.pdf}
        \captionsep{}
        \mycaption{Abbildung 5}{Beobachtete (Punkte) und durch unbeschränkte GAMLSS vorhergesagte (farbige Linien) Grundfläche (\(G\)) über Oberhöhe (\(h_{100}\)) bei Buche (links) und Fichte (rechts). Schwarze Linien verbinden Beobachtungen einer Parzelle. Farben deuten die jeweilige Ertragsklasse an.}
      \end{minipage}}}
\end{frame}

\subsection{Beschränkte GAMLSS \textendash{} Ergebnisse}

\begin{frame}[c]
  \only<\theFirstElement>{
    \centerline{
      \begin{minipage}{0.9\textwidth}
        \includegraphics[width=1.0\textwidth]{../../Graphics/Presentation/TermEffectsGAMLSSConstrained.pdf}
        \captionsep{}
        \mycaption{Abbildung 6}{Effekt der Entwicklungsstufenvariable (\texttt{h100.EKL.I}, oben) und der Bonitätsvariable (\texttt{SI.h100.diff.EKL.I}, unten) im beschränkten GAMLSS für Buche (links) und Fichte (rechts).  Gestrichelte orange Linien markieren den Standardfehler.  Senkrechte graue Striche repräsentieren beobachtete Variablenwerte.}
      \end{minipage}}}

  \only<\theSecondElement>{
    \centerline{
      \begin{minipage}{0.9\textwidth}
        \includegraphics[width=1.0\textwidth]{../../Graphics/Presentation/StandAgeBasalAreaObservationsGAMLSSConstrainedPredictions.pdf}
        \captionsep{}
        \mycaption{Abbildung 7}{Beobachtete (Punkte) und durch beschränkte GAMLSS vorhergesagte (farbige Linien) Grundfläche (\(G\)) über Alter bei Buche (links) und Fichte (rechts). Schwarze Linien verbinden Beobachtungen einer Parzelle. Farben deuten die jeweilige Ertragsklasse an.}
      \end{minipage}}}

  \only<\theThirdElement>{
    \centerline{
      \begin{minipage}{0.9\textwidth}
        \includegraphics[width=1.0\textwidth]{../../Graphics/Presentation/TopHeightBasalAreaObservationsGAMLSSConstrainedPredictions.pdf}
        \captionsep{}
        \mycaption{Abbildung 8}{Beobachtete (Punkte) und durch beschränkte GAMLSS vorhergesagte (farbige Linien) Grundfläche (\(G\)) über Oberhöhe (\(h_{100}\)) bei Buche (links) und Fichte (rechts). Schwarze Linien verbinden Beobachtungen einer Parzelle. Farben deuten die jeweilige Ertragsklasse an.}
      \end{minipage}}}
\end{frame}

%%% Local Variables:
%%% mode: latex
%%% TeX-master: "MasArPresentation.tex"
%%% End:
