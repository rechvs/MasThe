\section{Einleitung}

\subsubsection*{}
\begin{frame}[c]
  \only<\theFirstElement>{}
\end{frame}

\subsection{Ziele, Probleme, Lösungsansätze}
\begin{frame}[c]
  \begin{itemize}
  \item<\theFirstElement-> Ziel: \\
    Modellierung der \emph{maximalen} Bestandesgrundfläche in Abhängigkeit von \emph{Alter} und \emph{Bonität} gleichaltriger Buchen- und Fichtenreinbestände
  \item<\theSecondElement-> Probleme:
    \begin{itemize}
    \item<\theSecondElement-> Trainingdatensatz muss von maximalbestockten Flächen stammen
    \item<\theSecondElement-> Direkt messbare Größen (z.B. \(h_{100}\)) als Prädiktoren ungeeignet, da sie Alters- und Bonitätseffekte enthalten
    \item<\theSecondElement-> Gängige Modelle erscheinen unzureichend
    \end{itemize}
  \end{itemize}
  \begin{itemize}
  \item<\theThirdElement-> Lösungsansätze:
    \begin{itemize}
    \item<\theThirdElement-> Auswahl der Trainingdaten anhand der Mortalitätsrate
    \item<\theThirdElement-> Trennung von Alters- und Bonitätseffekten
    \item<\theThirdElement-> GAMs (Generalized Additive Models) und GAMLSSs (Generalized Additive Models for Location, Scale and Shape)
    \end{itemize}
  \end{itemize}
\end{frame}

%%% Local Variables:
%%% mode: latex
%%% TeX-master: "MasArPresentation.tex"
%%% End:
