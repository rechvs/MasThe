\section{Einleitung}
\subsection{Ziele, Probleme, Lösungsansätze}
\begin{frame}[c]
  % \only<1>{
    % \centerline{
      % \begin{minipage}{0.9\textwidth}
        % \includegraphics[width=1.0\textwidth]{Grafiken/IMG_1181_beschnitten} \\
        % \mycaption{Abbildung: Beispiel für das Landnutzungsschema der Untersuchungsflächen (hier: KUP und Grünland)
          % (Quelle: Falk Richter, verändert).}
      % \end{minipage}}}

  \begin{itemize}
  \item<\thefirstElement-> Ziel: \\
    Modellierung der \emph{maximalen} Bestandesgrundfläche in Abhängigkeit von \emph{Alter} und \emph{Bonität} gleichaltriger Buchen- und Fichtenreinbestände
  \item<\thesecondElement-> Probleme:
    \begin{itemize}
    \item<\thesecondElement-> Trainingdatensatz muss von maximalbestockten Flächen stammen
    \item<\thesecondElement-> Direkt messbare Größen (z.B. \(h_{100}\)) als Prädiktoren ungeeignet, da sie Alters- und Bonitätseffekte enthalten
    \item<\thesecondElement-> Gängige Modelle erscheinen unzureichend
    \end{itemize}
  \end{itemize}
  \begin{itemize}
  \item<\thethirdElement-> Lösungsansätze:
    \begin{itemize}
    \item<\thethirdElement-> Auswahl der Trainingdaten anhand der Mortalitätsrate
    \item<\thethirdElement-> Trennung von Alters- und Bonitätseffekten
    \item<\thethirdElement-> GAMs (Generalized Additive Models) und GAMLSSs (Generalized Additive Models for Location, Scale and Shape)
    \end{itemize}
  \end{itemize}
\end{frame}

%%% Local Variables:
%%% mode: latex
%%% TeX-master: "MasArPresentation.tex"
%%% End:
