
% \section*{}
% % \subsubsection*{}
% \begin{frame}[c]
%   \only<\theFirstElement>{}
% \end{frame}

\section{Einleitung}
\subsection{Ziel und Motivation}

\begin{frame}[c]
  \begin{itemize}
  \item<\theFirstElement-> Ziel: \\
    Modellierung der \emph{maximalen} Bestandesgrundfläche in Abhängigkeit von \emph{Entwicklungsstufe} und \emph{Bonität} gleichaltriger Buchen- und Fichtenreinbestände
  \item<\theFirstElement-> Motivation:
    \begin{itemize}
    \item Modellierung dichteabhängiger Mortalität
    \item Referenz für oberhöhengesteuerte Grundflächenhaltung
    \end{itemize}
  \end{itemize}
\end{frame}

\subsection{Probleme und Lösungsansätze}
\begin{frame}[c]
  \begin{itemize}
  \item<\theFirstElement-> Probleme:
    \begin{itemize}
    \item<\theFirstElement-> Trainingdatensatz muss von maximalbestockten Flächen stammen
    \item<\theSecondElement-> \(h_{100}\) als Prädiktor nicht ideal, da sie sowohl Entwicklungsstufen- als auch Bonitätseffekte enthält
    \item<\theThirdElement-> Bisherige Modelle für Zielerreichung nicht geeignet
    \end{itemize}
  \end{itemize}
  \begin{itemize}
  \item<\theFirstElement-> Lösungsansätze:
    \begin{itemize}
    \item<\theFirstElement-> Auswahl der Trainingdaten mithilfe der Reineke-Gleichung
    \item<\theSecondElement-> Trennung von Entwicklungsstufen- und Bonitätseffekten
    \item<\theThirdElement-> GAM (Generalized Additive Model) und \\
      GAMLSS (Generalized Additive Model for Location, Scale and Shape)
    \end{itemize}
  \end{itemize}
\end{frame}

%%% Local Variables:
%%% mode: latex
%%% TeX-master: "MasArPresentation.tex"
%%% End:
