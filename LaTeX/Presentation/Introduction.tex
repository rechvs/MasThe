
\section*{}
% \subsubsection*{}
\begin{frame}[c]
  \only<\theFirstElement>{}
\end{frame}

\section{Einleitung}
\subsection{Ziel, Probleme, Lösungsansätze}
\begin{frame}[c]
  \begin{itemize}
  \item<\theFirstElement-> Ziel: \\
    Modellierung der \emph{maximalen} Bestandesgrundfläche in Abhängigkeit von \emph{Bonität} und \emph{Alter} gleichaltriger Buchen- und Fichtenreinbestände
  \item<\theSecondElement-> Probleme:
    \begin{itemize}
    \item<\theSecondElement-> Trainingdatensatz muss von maximalbestockten Flächen stammen
    \item<\theSecondElement-> Direkt messbare Variablen (z.B. \(h_{100}\)) als Prädiktoren ungeeignet, da sie sowohl Bonitäts- als auch Alterseffekte enthalten
    \item<\theSecondElement-> Gängige Modelle erscheinen unzureichend
    \end{itemize}
  \end{itemize}
  \begin{itemize}
  \item<\theThirdElement-> Lösungsansätze:
    \begin{itemize}
    \item<\theThirdElement-> Auswahl der Trainingdaten anhand der Mortalitätsrate
    \item<\theThirdElement-> Trennung von Bonitäts- und Alterseffekten
    \item<\theThirdElement-> GAM (Generalized Additive Model) und GAMLSS (Generalized Additive Model for Location, Scale and Shape)
    \end{itemize}
  \end{itemize}
\end{frame}

%%% Local Variables:
%%% mode: latex
%%% TeX-master: "MasArPresentation.tex"
%%% End:
