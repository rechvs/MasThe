\section*{}

\begin{frame}[plain]
  \begin{center}
    \begin{minipage}{0.33\linewidth}
      \mycaption{Tabelle S1}{Für den Auswahlmechanismus verwendete Schwellwerte ,,erlaubter`` Steigungen.}
      \begin{tabular}{l l l}
        \toprule
        Art & \(m_u\) & \(m_o\) \\
        \midrule
        Buche & -2.91 & -0.9 \\
        Fichte & -2.82 & -0.65 \\
        \bottomrule
      \end{tabular}
    \end{minipage}
  \end{center}
  
  \begin{center}
    \begin{minipage}{0.32\linewidth}
      \mycaption{Tabelle S2}{In \VospernikSterba{} wiedergegebene Steigungen für \textbf{Gleichung 1}.}
      \begin{tabular}{l l l}
        \toprule
        Art & Min. & Max. \\
        \midrule
        Buche & -1.94 & -1.6 \\
        Fichte & -1.88 & -1.3 \\
        \bottomrule
      \end{tabular}
    \end{minipage}
  \end{center}

  \begin{center}
    \begin{minipage}{0.4\linewidth}
      {\scriptsize
        \mycaption{Tabelle S3}{Linkfunktionen für die Verteilungsparameter der \texttt{BCCGo()}-Verteilung.}
        \begin{tabular}{l l}
          \toprule
          Verteilungsparameter & Linkfunktion \\
          \midrule
          \(\mu\) & Logarithmus \\
          \(\sigma\) & Logarithmus \\
          \(\nu\) & Identität \\
          \bottomrule
        \end{tabular}}
    \end{minipage}
  \end{center}
\end{frame}

%%% Local Variables:
%%% mode: latex
%%% TeX-master: "MasArPresentation.tex"
%%% End:
