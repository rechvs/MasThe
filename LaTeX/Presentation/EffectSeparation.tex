\section{Trennung der Effekte}
\begin{frame}[c]

  \begin{itemize}
  \item<\theFirstElement-> Problem: \\
    \(h_{100}\) als Prädiktor ungeeignet, da sie (bei geringem Datenumfang) keine Trennung von Entwicklungsstufen- und Bonitätseffekt zulässt
  \item<\theFirstElement-> Lösungsansatz: \\
    Separate Berechung von Entwicklungsstufen- und Bonitätseffekten
  \end{itemize}

  % \begin{center}
  %   \begin{minipage}{0.75\linewidth}
  %     \centerline{\visible<\theFirstElement->{\NagelFunctionSolvedForSI{}}}
  %     \vspace{\captiondistance}
  %     \mycaption{Gleichung 2}{Absolute Bonität \(SI\) in Abhängigkeit von \(h_{100}\) und Alter (Quelle: [3]).}
  %   \end{minipage}
  % \end{center}

\end{frame}

\subsection{Entwicklungsstufeneffekt}

\begin{frame}[c]
  \visible<\theFirstElement->{Vorgehensweise:}
  \begin{enumerate}
  \item Berechnung der Oberhöhe \(h_{100}\) mittels \textbf{Gleichung 3}
  \end{enumerate}
  
    \visible<\theFirstElement->{\begin{center}
      \begin{minipage}{0.665\linewidth}
        \centerline{\(\begin{aligned}[t]\NagelFunctionSimpleSolvedForhSIIEKL{}\end{aligned}\)}
        \vspace{\captiondistance}
        \mycaption{Gleichung 3}{\(h_{100}\) in Abhängigkeit von absoluter Bonität \(SI\) und Alter, unterstellt der Bestand sei I. EKL (Quelle: [3]). \\
          \(h_{100}(x)\): Oberhöhe im Alter \(x\) [\si{\meter}]\\
          \(SI_{\text{I. EKL}}\): \(h_{100}\) im Alter \SI{100}{a} der I. EKL (Quelle: [4]) [\si{\meter}]\\
          \(\ln\): natürlicher Logarithmus \\
        \(\beta_0, \ldots, \beta_4\): Koeffizienten}
      \end{minipage}
    \end{center}}
\end{frame}

\subsection{Bonitätseffekt}

\begin{frame}[c]
  \visible<\theFirstElement->{Vorgehensweise:
  \begin{enumerate}
  \item Berechnung der absoluten Bonität \(SI\) mittels \textbf{Gleichung 2}
  \item Berechnung der Differenz zwischen 1. und \(SI_{\text{I. EKL}}\) (Quelle: [4])
  \end{enumerate}}

  \visible<\theFirstElement->{\begin{center}
      \begin{minipage}{0.5\linewidth}
        \centerline{\(\NagelFunctionSimpleSolvedForSI{}\)}
        \vspace{\captiondistance}
        \mycaption{Gleichung 2}{Absolute Bonität \(SI\) in Abhängigkeit von \(h_{100}\) und Alter (Quelle: [3]). \\
          \(SI\): \(h_{100}\) im Alter \SI{100}{a} [\si{\meter}]\\
          \(h_{100}(x)\): Oberhöhe im Alter \(x\) [\si{\meter}]\\
          \(\ln\): natürlicher Logarithmus \\
        \(\beta_0, \ldots, \beta_4\): Koeffizienten}
      \end{minipage}
    \end{center}}

  % \visible<\theSecondElement->{Beispiel:
  %   \begin{itemize}
  %     % \item Alter = \SI{91}{a}
  %   \item \(h_{100}(91) = \SI{29.4}{\meter}\) \\
  %     \(\Rightarrow SI = \SI{30.39}{\meter}\)
  %   \item \(SI_{\text{I. EKL}} = \SI{32.4}{\meter}\)
  %   \item \(\begin{aligned}[t]
  %       \Delta_{SI} &= \SI{30.39}{\meter} - \SI{32.4}{\meter} \\
  %       &= \SI{-2.01}{\meter} \\
  %     \end{aligned}\)
  %   \end{itemize}}
\end{frame}

%%% Local Variables:
%%% mode: latex
%%% TeX-master: "MasArPresentation.tex"
%%% End:
