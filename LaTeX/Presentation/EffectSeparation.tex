\section{Trennung der Effekte}
\begin{frame}[c]

  \begin{itemize}
  \item<\theFirstElement-> Problem: \\
    \(h_{100}\) als Prädiktor ungeeignet, da sie sowohl durch Alter als auch durch Bonität beeinflusst wird
  \item<\theFirstElement-> Lösungsansatz: \\
    Aufspaltung von \(h_{100}\) in Bonitäts- und Alterseffekte mittels
  \end{itemize}

  \begin{center}
    \begin{minipage}{0.75\linewidth}
      \centerline{\visible<\theFirstElement->{\NagelFunctionSolvedForSI{}}}
      \vspace{\captiondistance}
      \mycaption{Gleichung}{Absolute Bonität \(SI\) in Abhängigkeit von \(h_{100}\) und Alter (Quelle: [3]).}
    \end{minipage}
  \end{center}

\end{frame}

\subsection{Bonitätseffekt}

\begin{frame}[c]

  \visible<\theFirstElement->{Vorgehensweise:}
  \begin{enumerate}
  \item<\theFirstElement-> Berechnung der absoluten Bonität
  \item<\theFirstElement-> Berechnung der Differenz zwischen 1. und absoluter Bonität der I. EKL
  \item<\theFirstElement-> Verwendung von 2. als Prediktor
  \end{enumerate}

  \visible<\theSecondElement->{Beispiel:}
  \begin{itemize}
  % \item<\theSecondElement-> Alter = \SI{91}{a}
  \item<\theSecondElement-> \(h_{100}(91) = \SI{29.4}{\meter}\)
  \item<\theSecondElement-> \(SI = \SI{30.39}{\meter}\)
  \item<\theSecondElement-> \(SI_{\text{I. EKL}} = \SI{32.4}{\meter}\)
  \item<\theSecondElement-> \(\begin{aligned}[t]
                              \Delta_{SI} &= \SI{30.39}{\meter} - \SI{32.4}{\meter} \\
                                          &= \SI{-2.01}{\meter} \\
                            \end{aligned}\)
  \end{itemize}

\end{frame}

%%% Local Variables:
%%% mode: latex
%%% TeX-master: "MasArPresentation.tex"
%%% End:
