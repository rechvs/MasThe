\section{Trennung der Effekte}

\begin{frame}[c]
  \begin{itemize}
  \item<\theFirstElement-> Problem: \\
    Oberhöhe (\(h_{100}\)) als Prädiktor ungeeignet, da sie (bei geringem Datenumfang) keine Trennung von Entwicklungsstufen- und Bonitätseffekt zulässt
  \item<\theFirstElement-> Lösungsansatz: \\
    Separate Berechnung einer Entwicklungsstufen- und einer Bonitätsvariable
  \end{itemize}
\end{frame}

\subsection{Entwicklungsstufenvariable}

\begin{frame}[c]
  \visible<\theFirstElement->{Vorgehensweise: \\
  Berechnung der Oberhöhe (\(h_{100}\)) mittels \textbf{Gleichung 2}}
  
    \visible<\theFirstElement->{\begin{center}
      \begin{minipage}{0.665\linewidth}
        \centerline{\(\begin{aligned}[t]\NagelFunctionSimpleSolvedForhSIIEKL{}\end{aligned}\)}
        \vspace{\captiondistance}
        \mycaption{Gleichung 2}{Oberhöhe in Abhängigkeit von absoluter Bonität und Alter, unterstellt der Bestand sei I. Ertragsklasse (Quelle: \Nagel{}). \\
          \(h_{100}(x)\): Oberhöhe im Alter \(x\) [\si{\meter}]\\
          \(SI_{\text{I. EKL}}\): \(h_{100}\) im Alter \SI{100}{a} der I. Ertragsklasse [\si{\meter}] (Quelle: \Schober{}) \\
          \(\ln\): natürlicher Logarithmus \\
        \(\beta_0, \ldots, \beta_4\): Koeffizienten}
      \end{minipage}
    \end{center}}
\end{frame}

\subsection{Bonitätsvariable}

\begin{frame}[c]
  \visible<\theFirstElement->{Vorgehensweise:
  \begin{enumerate}
  \item Berechnung der absoluten Bonität (\(SI\)) mittels \textbf{Gleichung 3}
  \item Berechnung der Differenz zwischen 1. und \(SI_{\text{I. EKL}}\) (Quelle: \Schober{})
  \end{enumerate}}

  \visible<\theFirstElement->{\begin{center}
      \begin{minipage}{0.5\linewidth}
        \centerline{\(\NagelFunctionSimpleSolvedForSI{}\)}
        \vspace{\captiondistance}
        \mycaption{Gleichung 3}{Absolute Bonität in Abhängigkeit von Oberhöhe und Alter (Quelle: \Nagel{}). \\
          \(SI\): \(h_{100}\) im Alter \SI{100}{a} [\si{\meter}]\\
          \(h_{100}(x)\): Oberhöhe im Alter \(x\) [\si{\meter}]\\
          \(\ln\): natürlicher Logarithmus \\
        \(\beta_0, \ldots, \beta_4\): Koeffizienten}
      \end{minipage}
    \end{center}}
\end{frame}

%%% Local Variables:
%%% mode: latex
%%% TeX-master: "MasArPresentation.tex"
%%% End:
