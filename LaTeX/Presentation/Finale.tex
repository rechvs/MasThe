\section*{}

\begin{frame}[plain]
  \begin{center}
    \setbeamercolor{crane}{fg=letters}
    \setbeamercolor{crane}{bg=boxes}

    \begin{minipage}{0.75\textwidth}
      \begin{beamercolorbox}{crane}
        \begin{center}
          \vspace{1em}
          \textbf{\huge Vielen Dank}
          \\
          \textbf{\small für die Aufmerksamkeit.}
          \\
          \vspace{1em}
        \end{center}
      \end{beamercolorbox}
    \end{minipage}

    \begin{minipage}{0.73\textwidth}  %% DO NOT change the minipage width, since it was manually set to ensure correct horizontal alignment of the previous and this minipage. 
      \setbeamertemplate{enumerate item}{[\insertenumlabel]}
      \begin{block}{Quellen}
        \begin{tiny}
          \begin{enumerate}
          \item \emph{Reineke L. H.} (1933): Perfecting a Stand-Density Index for even-aged Forests. \emph{Journal of Agricultural Research}, 46: 627--638
          \item \emph{Vospernik S., Sterba H.} (2015): Do competition-density rule and self-thinning rule agree? \emph{Annals of Forest Science}, 72: 379--390
          \item \emph{Nagel J.} (1999): Konzeptionelle Überlegungen zum schrittweisen Aufbau eines waldwachstumskundlichen Simulationssystems für Nordwestdeutschland. \emph{Schriften aus der Forstlichen Fakultät der Universität Göttingen und der Niedersächsischen Forstlichen Versuchsanstalt}, 128
          \item \emph{Schober R.} (1995): Ertragstafeln wichtiger Baumarten, Frankfurt am Main
          \end{enumerate}
        \end{tiny}
      \end{block}
    \end{minipage}
  \end{center}
\end{frame}

%%% Local Variables:
%%% mode: latex
%%% TeX-master: "MasArPresentation.tex"
%%% End:
