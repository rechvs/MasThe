\subsection{Models overview}

The logarithm was chosen as the link function in all models in order to prevent unrealistic predictions of negative basal area.

Following \textcite{Baur1881} (cited in \textcite[p. 159]{Assmann1970}), this study rests on the so-called “site index hypothesis” \parencite{Skovsgaard2008}, which may be broken down into 2 assumptions:
\begin{enumerate}
\item Yield classes can be sufficiently distinguished based on stand top height at a given reference age, i.e., their site index.
\item Stands of different yield classes attain different basal areas at a given age, while stands of the same yield class attain the same basal area at a given age.
\end{enumerate}
These assumptions are based on the belief that top height in pure even-aged unthinned stands is largely independent of stem number \parencite{Skovsgaard2008}.  
They are, however, are not undisputed.  \textcite{Assmann1970} pointed out that stands sharing the same site index still show site-dependent differences in total crop yield as well as basal area.  Nevertheless, site index continues to be a subject of research \parencite{Weiskittel2011,Somarriba2001,Wang2005} and was considered an indicator of site productivity by others \parencite{Monserud1984,Rayner1992,Karlsson1997}.  The goal of this study was not to determine whether or not site index is an appropriate measure for classfying site productivity, but rather to develope tools which allow modeling of maximum basal area while also taking differences of site productivity into account.  Whether these differences are expressed in terms of site index or by other means is of no relevance in this regard.



%%% Local Variables:
%%% mode: latex
%%% TeX-master: "MasArThesis.tex"
%%% End:
