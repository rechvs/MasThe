\subsection{GAM1}

% The results of model GAM1 are reported in \Cref{fig:GAM1EffectStandAgeVariable}.
\Cref{fig:GAM1EffectStandAgeVariable} depicts the estimated effect of the stand age variable smooth function over the stand age variable for both species in model GAM1.
% In both plots, the confidence band does include non-zero values of the estimate.  This suggests that the stand age variable is related to basal area in both species \parencite{Wood2001}.
Due to the fact that the \Beech{} data set does not contain any observations for \(\StandAgeVariableMath{} \leq \SI{12}{\meter}\), standard errors of the estimate increase notably below this value in the top plot.  The \Spruce{} plot exhibits a widening of its confidence bands only at the edges of the depicted interval.
For \Beech{}, the estimated effect reaches its absolute maximum at roughly \(\StandAgeVariableMath{} = \SI{27.5}{\meter}\), which corresponds to an age of about \SI{76}{\year}.  After that, the effect starts to decrease.  In the case of \Spruce{}, the effect increases throughout all of the interval depicted, leading to the highest estimate being at the end of the interval depicted (\(\StandAgeVariableMath{} = \SI{40}{\meter}\)), which corresponds to an age of about \SI{125}{\year}.  Compared to \Spruce{}, the estimated effect of the stand age variable covers a narrower range in the \Beech{} model.  This is in accordance with differences in the range of the stand age variable between species.  The estimated effect does not show signs of asymptotic behavior in either species.  The \edf{} of the smooth functions were \num{2.66} and \num{6.39} for \Beech{} and \Spruce{}, respectively.

\Cref{fig:GAM1EffectProductivityIndexVariable} depicts the estimated effect of the \ProductivityIndexVariableText{} smooth function over the \ProductivityIndexVariableText{} for both species in model GAM1.
% In both plots, the confidence band does include non-zero values of the estimate.  This suggests that the stand age variable is related to basal area in both species \parencite{Wood2001}.
The \Beech{} data set covers a narrower range of the \ProductivityIndexVariableText{}.  Consequently, the confidence bands start to widen further from the edges in the top plot compared to the bottom plot.  In the case of \Beech{}, the estimated effect shows a sharp increase for \(\SI{-12.5}{\meter} \leq \ProductivityIndexVariableMath{} \leq \SI{-2.5}{\meter}\), starting at an estimate of approximately \num{3} before almost completely levelling off at an estimate of zero.  In the case of \Spruce{}, the estimate starts at around zero and never strays very far from it in the interval depicted.  Nevertheless, the confidence band does include non-zero estimates in both plots, suggesting that the \ProductivityIndexVariableText{} is related to basal area in both species \parencite{Wood2001}.




\begin{figure}[h]
  \centering
  \includegraphics[width=1.0\textwidth]{../../Graphics/Thesis/GAM1EffectStandAgeVariable.pdf}
  \caption{Estimated effect of the stand age variable smooth function (\(s(\StandAgeVariableMath{}, \ldots)\)) over stand age variable (\(\StandAgeVariableMath{}\)) in model GAM1 for \Beech{} (top) and \Spruce{} (bottom).  Solid lines mark estimates.  Dashed lines mark confidence bands of 2 standard errors width.  Vertical bars mark observed values.  Numbers in the y-axis titles are the effective degrees of freedom of the smooth function.}
  \label{fig:GAM1EffectStandAgeVariable}
\end{figure}

\begin{figure}[h]
  \centering
  \includegraphics[width=1.0\textwidth]{../../Graphics/Thesis/GAM1EffectProductivityIndexVariable.pdf}
  \caption{Estimated effect of the \ProductivityIndexVariableText{} smooth function (\(s(\ProductivityIndexVariableMath{}, \ldots)\)) over \ProductivityIndexVariableText{} (\(\ProductivityIndexVariableMath{}\)) in model GAM1 for \Beech{} (top) and \Spruce{} (bottom).  Solid lines, dashed lines, vertical bars, and numbers in the y-axis titles have the same meaning as in \Cref{fig:GAM1EffectStandAgeVariable}, namely:  Solid lines mark estimates.  Dashed lines mark confidence bands of 2 standard errors width.  Vertical bars mark observed values.  Numbers in the y-axis titles are the effective degrees of freedom of the smooth function.}
  \label{fig:GAM1EffectProductivityIndexVariable}
\end{figure}



\clearpage{}

%%% Local Variables:
%%% mode: latex
%%% TeX-master: "MasArThesis.tex"
%%% End:
