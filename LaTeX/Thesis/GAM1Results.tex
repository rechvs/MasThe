\subsection{GAM1}

% The results of model GAM1 are reported in \Cref{fig:GAM1EffectStandAgeVariable}.
\Cref{fig:GAM1EffectStandAgeVariable} depicts the estimated effect of the stand age variable smooth function over the stand age variable for both species in model GAM1.
% In both plots, the confidence band does include non-zero values of the estimate.  This suggests that the stand age variable is related to basal area in both species \parencite{Wood2001}.
Due to the fact that the \Beech{} data set does not contain any observations for \(\StandAgeVariableMath{} \leq \SI{12}{\meter}\), standard errors of the estimate increase notably below this value in the top plot.  The \Spruce{} plot exhibits a widening of its confidence bands only at the edges of the depicted interval.
For \Beech{}, the estimated effect reaches its absolute maximum at roughly \(\StandAgeVariableMath{} = \SI{27.5}{\meter}\), which corresponds to an age of about \SI{76}{\year}.  After that, the effect starts to decrease.  In the case of \Spruce{}, the effect increases throughout all of the interval depicted, leading to the highest estimate being at the end of the interval depicted (\(\StandAgeVariableMath{} = \SI{40}{\meter}\)), which corresponds to an age of about \SI{125}{\year}.  Compared to \Spruce{}, the estimated effect of the stand age variable covers a narrower range in the \Beech{} model.  This is in accordance with differences in the range of the stand age variable between species.  The estimated effect does not show signs of asymptotic behavior in either species.  The \edf{} of the smooth functions were \num{2.66} and \num{6.39} for \Beech{} and \Spruce{}, respectively.

\Cref{fig:GAM1EffectProductivityIndexVariable} depicts the estimated effect of the \ProductivityIndexVariableText{} smooth function over the \ProductivityIndexVariableText{} for both species in model GAM1.
% In both plots, the confidence band does include non-zero values of the estimate.  This suggests that the stand age variable is related to basal area in both species \parencite{Wood2001}.
The \Beech{} data set covers a narrower range of the \ProductivityIndexVariableText{}.  Consequently, the confidence bands start to widen further from the edges in the top plot compared to the bottom plot.  In the case of \Beech{}, the estimated effect shows a sharp increase for \(\SI{-12.5}{\meter} \leq \ProductivityIndexVariableMath{} \leq \SI{-2.5}{\meter}\), starting at an estimate of approximately \num{3} before almost completely levelling off at an estimate of zero.  In the case of \Spruce{}, the estimate starts at around zero and never strays very far from it in the interval depicted.  Nevertheless, the confidence band does include non-zero estimates in both plots, suggesting that the \ProductivityIndexVariableText{} is related to basal area in both species \parencite{Wood2001}.

\Cref{fig:GAM1QQPlot} shows the quantile-quantile plot for model GAM1 for both species.  In both species, response residuals follow the 1:1 reference line fairly closely with some residuals lying outside the \SI{90}{\percent} reference band.

\Cref{fig:GAM1LinearPredictorResiduals} shows the response residuals over link the function value for the corresponding prediction of model GAM1 for both species.  In \Beech{}, variability of residuals does increase notably as link function value increases.  In \Spruce{}, the same trend can be observed albeit less pronounced.

\Cref{fig:GAM1StandAgeBasalAreaObservationsPredictionsYieldClassClassification} shows basal area over stand age, both observed values as well as predictions of model GAM1.  General behavior of the model is notably different for each species:  in the case of \Beech{}, predicted basal area starts to decrease between stand ages of \SIrange{70}{90}{\year}, while in \Spruce{} it increases throughout the whole depicted stand age interval, especially between \SIrange{80}{100}{\year}.  In both species, predicted basal area shows stratification depending on yield class.  However, the order of curves does not coincide with yield class order:  in the case of \Beech{}, yield class 1 lies above yield class 0, while in the case of \Spruce{}, yield class 1 lies above yield class -2 and yield class 3 lies above yield class 2.  In \Beech{}, yield class 3 performs much worse than yield class 2.  Other yield classes of \Beech{} and all yield classes of \Spruce{} do not exhibit as large distances between adjacent yield classes. 

\Cref{fig:GAM1TopHeightBasalAreaObservationsPredictionsYieldClassClassification} shows basal area over top height, both observed values as well as predictions of model GAM1.  General model behavior is the same as in \Cref{fig:GAM1StandAgeBasalAreaObservationsPredictionsYieldClassClassification}.  Stratification of curves depending on yield class is visible.  In the case of \Beech{}, the order of curves (excluding yield class 3) changes during the depicted age interval.  For top heights up to \SI{22}{\meter}, order of curves is almost the exact inverse of yield class order, with yield class 2 showing best performance and yield class -2 showing worst performance.  Between top heights of \SIrange{22}{35}{\meter}, order of curves shifts until eventually stratification follows yield class order.  In contrast, \Spruce{} shows a highly erratic pattern of stratification, the only constant of which is that yield class -2 performs worst of all yield classes throughout.

\begin{figure}[h]
  \centering
  \includegraphics[width=1.0\textwidth]{../../Graphics/Thesis/GAM1EffectStandAgeVariable.pdf}
  \caption{Estimated effect of the stand age variable smooth function (\(s(\StandAgeVariableMath{}, \ldots)\)) over stand age variable (\(\StandAgeVariableMath{}\)) in model GAM1 for \Beech{} (top) and \Spruce{} (bottom).  Solid lines mark estimates.  Dashed lines mark confidence bands of 2 standard errors width.  Vertical bars mark observed values.  Numbers in the y-axis titles are the effective degrees of freedom of the smooth function.}
  \label{fig:GAM1EffectStandAgeVariable}
\end{figure}

\begin{figure}[h]
  \centering
  \includegraphics[width=1.0\textwidth]{../../Graphics/Thesis/GAM1EffectProductivityIndexVariable.pdf}
  \caption{Estimated effect of the \ProductivityIndexVariableText{} smooth function (\(s(\ProductivityIndexVariableMath{}, \ldots)\)) over \ProductivityIndexVariableText{} (\(\ProductivityIndexVariableMath{}\)) in model GAM1 for \Beech{} (top) and \Spruce{} (bottom).  Solid lines, dashed lines, vertical bars, and numbers in the y-axis titles have the same meaning as in \Cref{fig:GAM1EffectStandAgeVariable}, namely:  Solid lines mark estimates.  Dashed lines mark confidence bands of 2 standard errors width.  Vertical bars mark observed values.  Numbers in the y-axis titles are the effective degrees of freedom of the smooth function.}
  \label{fig:GAM1EffectProductivityIndexVariable}
\end{figure}

\begin{figure}[h]
  \centering
  \includegraphics[width=1.0\textwidth]{../../Graphics/Thesis/GAM1QQPlot.pdf}
  \caption{Quantile-quantile plot of observed values minus predicted values (response residuals) over corresponding prediction quantiles (theoretical quantiles) of model GAM1 for \Beech{} (top) and \Spruce{} (bottom).  Black lines are 1:1 reference lines.  Black dots are residuals.  Gray shaded areas mark reference bands between the \num{0.05} and \num{0.95} quantiles of predictions (\SI{90}{\percent} level).  Theoretical quantiles and reference bands are based on repeated predictions (\(N = \num{e4}\)) \parencite{Augustin2012}.  Note the different axis scaling in both plots.}
  \label{fig:GAM1QQPlot}
\end{figure}

\begin{figure}[h]
  \centering
  \includegraphics[width=1.0\textwidth]{../../Graphics/Thesis/GAM1LinearPredictorResiduals.pdf}
  \caption{Observed values minus predicted values (response residuals) over value of the link function applied to the predicted values (linear predictor) of model GAM1 for \Beech{} (top) and \Spruce{} (bottom).  Note the different axis scaling in both plots.}
  \label{fig:GAM1LinearPredictorResiduals}
\end{figure}

\begin{figure}[h]
  \centering
  \includegraphics[width=1.0\textwidth]{../../Graphics/Thesis/GAM1StandAgeBasalAreaObservationsPredictionsYieldClassClassification.pdf}
  \caption{Basal area (\(G\)) over stand age of \Beech{} (top) and \Spruce{} (bottom).  Colored lines represent predictions of model GAM1.  Dots, black lines, and colors have the same meaning as in \Cref{fig:StandAgeTopHeightYieldClassClassification}, namely:  Each dot represents one observation.  Black lines connect observations belonging to the same sample plot.  Color signifies the (fractional) yield class of the respective observation or prediction, ranging from red (worst yield class observed) over yellow and green to blue (best yield class observed). Yield class classification was based on \textcite{Schober1995} (moderate thinning).  Note the different yield class ranges in both plots.}
  \label{fig:GAM1StandAgeBasalAreaObservationsPredictionsYieldClassClassification}
\end{figure}

\begin{figure}[h]
  \centering
  \includegraphics[width=1.0\textwidth]{../../Graphics/Thesis/GAM1TopHeightBasalAreaObservationsPredictionsYieldClassClassification.pdf}
  \caption{Basal area (\(G\)) over top height (\(\TopHeight{}\)) of \Beech{} (top) and \Spruce{} (bottom).  Colored lines, dots, black lines, and colors have the same meaning as in \Cref{fig:GAM1StandAgeBasalAreaObservationsPredictionsYieldClassClassification}, namely:  Colored lines represent predictions of model GAM1.  Each dot represents one observation.  Black lines connect observations belonging to the same sample plot.  Color signifies the (fractional) yield class of the respective observation or prediction, ranging from red (worst yield class observed) over yellow and green to blue (best yield class observed). Yield class classification was based on \textcite{Schober1995} (moderate thinning).  Note the different yield class ranges in both plots.}
  \label{fig:GAM1TopHeightBasalAreaObservationsPredictionsYieldClassClassification}
\end{figure}


\clearpage{}

%%% Local Variables:
%%% mode: latex
%%% TeX-master: "MasArThesis.tex"
%%% End:
