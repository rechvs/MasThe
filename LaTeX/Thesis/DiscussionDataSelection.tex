\subsection{Data selection}

Both the lower and the upper slope thresholds used in the data selection mechanism (cp. \Cref{tab:ReinekeSlopeThresholds}) differ notably from the species-specific slopes of \Cref{eq:Reineke} as reported in the literature (cp. \Cref{tab:SpeciesSpecificReinekeSlopes}).  However, only the maximum of observed slopes was affected by the data selection mechanism, whereas the minimum remained the same before and after its application in both species (cp. \Cref{tab:ObservedReinekeSlopes}).  The minimum was close to, albeit lower than, the slopes reported in the literature.  In contrast, the maximum slope, both before and after application of the mechanism, notably exceeded even the highest slopes reported in the literature in both species.   This suggests that the upper slope threshold may have been too high in order to reliably exclude all observations of stands which were not (yet) subject to self-thinning, which in turn suggests that the predictions reported may underestimate the maximum basal area possible on sites similar to the sample plots.  Lowering the upper thresholds to the highest values reported in the literature would have reduced the data set size for \Beech{} and \Spruce{} to \num{6} observations and \num{26} observations, respectively.  Since using such small data sets for model fitting would not have yielded generalizable models, lowering the upper slope thresholds was not an option for increasing data set size.

%%% Local Variables:
%%% mode: latex
%%% TeX-master: "MasArThesis.tex"
%%% End:
