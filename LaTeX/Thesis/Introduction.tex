The term ``maximum basal area'' refers to the site-specific highest possible basal area of living trees in a forst stand \parencite{Assmann1970}.  It plays a key role in forest management and planning, e.g., as a reference for determining thinning intensity \parencite{Assmann1961,Doebbeler2002} or stand density \parencite{Spellmann1999} or as an estimator of yield level \parencite{Franz1967}.  At the same time, permanent sample plots on which maximum basal area is recorded are few and far between and cover only a narrow selection of stand properties.  Thus, the need for extrapolating maximum basal area for stands for which not comparable experimental plot exists arises.

Basal area modeling in general has been approached from various directions, both via remote sensing imagery (e.g., \textcite{Silva2017}) as well as using in situ measurements (e.g., \textcite{Yue2012}).  Models may either be sensitive to stand productivity (e.g., \textcite{Castedo-Dorado2007}) or insensitive (e.g., \textcite{Monserud1996}).  However, only few models are available for predicting \emph{maximum} basal area, with the method proposed by \textcite{Sterba1975} being a widely used example.  Recently, \textcite{Woerdehoff2014,Woerdehoff2016} extended the range of available methods by utilizing generalized additive models for location, scale, and shape (GAMLSSs, \textcite{Rigby2001}), which are an extension of generalized additive models (GAMS, \textcite{Hastie1991}).  However, the available methods only use top height as a predictor variable and do not take stand productivity into account.  ``Stand productivity'' here means the total crop yield achieved for a given top height \parencite{Assmann1970}.  The present study aims to fill this gap and provide an age- and productivity-sensitive method for predicting maximum basal area.  Extending the method proposed by \textcite{Woerdehoff2016}, it uses GAMs, shape constrained additive models (SCAMs, which are an extension of GAMS \parencite{Pya2010}), and GAMLSSs with varying constraints.  Using smooth functions and splines to construct their function basis, these models provide greater flexibility than Generalized Linear Models (GLMs, \textcite{Nelder1972}), albeit at the price of a less intuitive interpretation.  In order to achieve age- and productivity-sensitivity of the models, suited stand properties need to be used as predictor variables.  Past experience has shown that usage of top height as a predictor variable prohibits the inclusion of a distinct productivity-effect.  Building on the work by \textcite{Nagel1999}, this study attempts to split up the information contained in top height into an age variable and a productivity variable, which may then be used as predictor variables in model fitting.

In order to increase reliability of model predictions to be \emph{maximum} basal area predictions, it is necessary to ensure that the training data used for model fitting belong to stands which have attained maximum basal area.  The most reliable way to achieve this is through continued measurements on permanent sample plots which ideally undergo no thinning apart from the removal of dead trees.  Unfortunately, such sample plots are not as numerous as might be required to obtain broadly applicable models.  As an alternative, observations of stands which are approximately at maximum basal area may be selected from readily available data sets.  Basal area itself is not a valid selection criterion in this regard and therefore has to be replaced by a closely related variable.  The mechanism presented here uses the slope of the equation proposed by \textcite{Reineke1933} as a criterion, by comparing observed values to a range of species-specific values of stands undergoing self-thinning.

6 models were fitted, 2 GAMs, one SCAM, and 3 GAMLSSs.  The models differ in the degree of constraints imposed on them.  Each model was fitted to a \BeechLong{} (\Beech{}) and a \SpruceLong{} (\Spruce{}) data set.  The data sets consisted of even-aged pure stands of the respective species.  The data sets used were kindly provided by the \NWFVA{}.

% First, the data selection mechanism will be explained.  Then, the resulting data sets will be described, with emphasis on the method used for classifying observations according to yield class.  Subsequently, the method for calculating the age and productivity variable will be explained.  In the following section, the mathematical background of GAMs, SCAMs, and GAMLSSs will be examined.  Afterwards, the method used for generating the test data will be detailed.  

%%% Local Variables:
%%% mode: latex
%%% TeX-master: "MasArThesis.tex"
%%% End:
