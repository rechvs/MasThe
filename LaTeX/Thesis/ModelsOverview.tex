\subsection{Model overview}

\RefTab{tab:PresentedModelsOverviewFormulas} provides an overview of the settings used for fitting the models presented in this study.  For each species, 6 different models were fitted: 2 GAMs, one SCAM, and 3 GAMLSSs.  The settings for a given model were the same for \Beech{} and \Spruce{}.  In the case of GAM1 and GAM2, the \texttt{s(\textnormal{\ldots})} formula terms are smooth functions using thin plate regression splines as their function basis.  In the case of SCAM1, the \texttt{s(\textnormal{\ldots}, bs = "micv")} formula term is a smooth function using P-splines constrained to be increasing and concave as its function basis.  In all smooth functions of the GAMS and SCAM, the setting of basis dimension and order of penalty is left to the smooth function.  In the case of GAMLSS1 and GAMLSS2, the \texttt{ps(\textnormal{\ldots})} formula terms are smooth functions using P-splines as their function basis.  In the case of GAMLSS3, the \texttt{pbm(\ldots{})} term is a smooth function using P-splines constrained to be montonone increasing as its function basis.  In all smooth functions of the 3 GAMLSSs, function basis degree was set to 3, function basis order was set to 2, the number of spline knots was set to 20, while selection of the smoothing parameter was left to the smooth function. For all GAMLSSs, the formula reported in \RefTab{tab:PresentedModelsOverviewFormulas} applies only to the location parameter of the assumed probability distribution. All other distribution parameters were modelled as constants (i.e., with formula \texttt{gha \textasciitilde{} 1}).  None of the 6 models assume interaction between the predictor variables.

\begin{table}[H]
  {\tabulinesep=2mm
    \begin{longtabu}{l l l L}
      \caption{Overview of the \texttt{R} functions and formulas used for fitting the models presented in this study. The overview includes
        the model ID,
        the name of the \texttt{R} package (and its version number) which provided the model fitting function,
        the name of the \texttt{R} model fitting function,
        and the formula used in the model fitting function call.
        In the case of GAMLSS1, GAMLSS2, and GAMLSS3, the formula applies only to the location parameter of the assumed probability distribution.
        For all other distribution parameters, the formula was \texttt{gha \textasciitilde{} 1}. \\
        \texttt{gha}: basal area variable \\
        \StandAgeVariableR{}: stand age variable (cp. \RefEq{eq:StandAgeVariable}) \\
        \ProductivityIndexVariableR{}: productivity index variable (cp. \RefEq{eq:SiteClassVariable})
        \label{tab:PresentedModelsOverviewFormulas}} \\
      \toprule
      Model ID & Package (Version) & Fitting function & Formula \\
      \midrule
      \endhead
      \bottomrule
      \endlastfoot
      GAM1 & \texttt{mgcv} (1.8.22) & \texttt{gam} & \texttt{gha \textasciitilde{} s(\StandAgeVariableR{}) + s(\ProductivityIndexVariableR{})} \\
      GAM2 & As above & As above & \texttt{gha \textasciitilde{} s(\StandAgeVariableR{}) + \ProductivityIndexVariableR{}} \\
      SCAM1 & \texttt{scam} (1.2.2) & \texttt{scam} & \texttt{gha \textasciitilde{} s(\StandAgeVariableR{}, bs = "micv") + \ProductivityIndexVariableR{}} \\
      GAMLSS1 & \texttt{gamlss} (5.0.4) & \texttt{gamlss} & \texttt{gha \textasciitilde{} ps(\StandAgeVariableR{}) + ps(\ProductivityIndexVariableR{})} \\
      GAMLSS2 & As above & As above & \texttt{gha \textasciitilde{} ps(\StandAgeVariableR{}) + \ProductivityIndexVariableR{}} \\
      GAMLSS3 & As above & As above & \texttt{gha \textasciitilde{} pbm(\StandAgeVariableR{}) + \ProductivityIndexVariableR{}} \\
      \bottomrule
    \end{longtabu}}
\end{table}

\RefTab{tab:PresentedModelsOverviewDistributions} provides an overview of the probability distributions and link functions employed in the presented models.  Models GAM1, GAM2, and SCAM1 assume basal area to follow Gamma distribution.  The probability density function of a variable following Gamma distribution is given by
\begin{equation}
  \label{eq:GammaDistributionPDF}
  P(X = x|k, \theta) = x^{k - 1} \frac{\exp{\Bigl(\frac{-x}{\theta}\Bigr)}}{\theta^k \upGamma(k)} \quad \text{for } x \geq 0, ~ k, \theta > 0
\end{equation}
while its  cumulative distribution function is given by
\begin{equation}
  \label{eq:GammaDistributionCDF}
  \begin{aligned}[b]
    D(X = x|k, \theta) &= P(X \leq x|k, \theta) \\
    &= 1 - \frac{\upGamma_{\text{i}}\Bigl(k, \frac{x}{\theta}\Bigr)}{\upGamma(k)} \quad \text{for } x \geq 0, ~ k, \theta > 0,
  \end{aligned}
\end{equation}
where \(k\) and \(\theta\) are the shape and scale parameter, respectively, \(\upGamma\) is the complete Gamma function, and \(\upGamma_{\text{i}}\) is the incomplete Gamma function \parencite{Weisstein2017b,Dormann2013,Lindgren1976}.  To avoid implausible negative predictions of basal area, the logarithm function was chosen as the link function, rather than the default inverse function.  Models GAMLSS1, GAMLSS2, and GAMLSS3 assume basal area to follow Box-Cox-Cole-Green distribution. This distribution is the Box-Cox transformation model presented by \textcite{Cole1992} \parencite{Stasinopoulos2007}.  Based on the transformations suggested by \textcite{Box1964}, the model first transforms the dependent variable \(y\) such that
\begin{equation}
  \label{eq:BoxCoxColeGreenTransformation}
  x =
  \begin{cases}
    \frac{
      \bigl(\left.
        y \middle/ \mu
      \right.\bigr)^\nu - 1}{\nu} &\text{if } \nu \not = 0 \\
    \log\bigl(\left.y \middle/ \mu\right.\bigr) &\text{if } \nu = 0,
  \end{cases}
\end{equation}
where \(x\) is the transformed variable, \(\mu\) is the mean of \(y\),
% \(\sigma\) is the coefficient of variation of \(y\),
and \(\nu\) is the Box-Cox power of \(y\) \parencite{Cole1992}.
The standard deviation score of \(x\) is given by
\begin{equation}
  \label{eq:BoxCoxColeGreenSDScore}
  z = \frac{x}{\sigma},
\end{equation}
where \(z\) is the standard deviation score of \(x\) and \(\sigma\) is the coefficient of variation of \(y\) .  It is assumed that \(z\) follows a normal distribution with mean 0 and variance 1. Thus, the probability density function of \(z\) is given by
\begin{equation}
  \label{eq:StandardNormalDistributionPDF}
  P(Z = z|0, 1) =
  \frac{\exp{\Bigl(-\frac{z^2}{2}\Bigr)}}{\sqrt{2 \pi}}
\end{equation}
while its cumulative distribution function is given by
\begin{equation}
  \label{eq:StandardNormalDistributionCDF}
  D(Z = z|0, 1) =
  \frac{\erf\biggl(\frac{z}{\sqrt{2}}\biggr) + 1}{2}
  \end{equation}
\parencite{Henze2013,Weisstein2017c}.  Substituting \(z\) in \Cref{eq:StandardNormalDistributionPDF,eq:StandardNormalDistributionCDF} using \Cref{eq:BoxCoxColeGreenTransformation,eq:BoxCoxColeGreenSDScore} yields the probabilty density function of \(y\)
\begin{equation}
  \label{eq:BoxCoxColeGreenDistributionPDF}
  P(Y = y|\mu, \sigma, \nu) =
  \begin{cases}
    \frac{\exp\Biggl(-\frac{\bigl(\bigl(\left.y \middle/ \mu\right.\bigr)^\nu - 1\bigr)^2}{2 \nu^2 \sigma^2}\Biggr)}{\sqrt{2 \pi}} &\text{if } \nu \not = 0 \\
    \frac{\exp\Biggl(-\frac{\log\bigl(\left.y \middle/ \mu\right.\bigr)^2}{2 \sigma^2}\Biggr)}{\sqrt{2 \pi}} &\text{if } \nu = 0
  \end{cases}
\end{equation}
and the cumulative distribution function of \(y\)
\begin{equation}
  \label{eq:BoxCoxColeGreenDistributionCDF}
  D(Y = y|\mu, \sigma, \nu) =
  \begin{cases}
    \frac{1}{2} \left(\erf\left(\frac{\left(\left. y \middle/ \mu \right.\right)^\nu - 1}{\sqrt{2} \nu \sigma}\right) + 1\right) &\text{if } \nu \not = 0 \vspace{0.5em} \\
    \frac{1}{2} \left(\erf\left(\frac{\log\left(\left. y \middle/ \mu \right.\right)}{\sqrt{2} \sigma}\right) + 1\right) &\text{if } \nu = 0.
  \end{cases}
\end{equation}

\begin{table}[H]
  {\tabulinesep=2mm
    \begin{longtabu}{l l L}
      \caption{Overview of the \texttt{R} distribution functions used for the models presented in this study.  The overview includes
        the model ID,
        the name of the \texttt{R} package (and its version number) which provided the distribution function,
        and the \texttt{R} call of the distribution function used in model fitting.
        \label{tab:PresentedModelsOverviewDistributions}} \\
      \toprule
      Model ID & Package (Version) & Distribution function call \\
      \midrule
      \endhead
      \bottomrule
      \endlastfoot
      GAM1 & \texttt{stats} (3.3.3) & \texttt{Gamma(link = "log")} \\
      GAM2 & As above & As above \\
      SCAM1 & As above & As above \\
      GAMLSS1 & \texttt{gamlss.dist} (5.0.3) & \texttt{BCCGo()} \\
      GAMLSS2 & As above & As above \\
      GAMLSS3 & As above & As above \\
      \bottomrule
    \end{longtabu}}
\end{table}

%%% Local Variables:
%%% mode: latex
%%% TeX-master: "MasArThesis.tex"
%%% End:
