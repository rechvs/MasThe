\subsection{Model overview}

\RefTab{tab:PresentedModelsOverviewFormulas} provides an overview of the settings used for fitting the models presented in this study.  For each species, 6 different models were fitted: 2 GAMs, one SCAM, and 3 GAMLSSs.  The settings for a given model were the same for both \Beech{} and \Spruce{}.

\begin{table}[H]
  {\tabulinesep=2mm
    \begin{longtabu}{l l l L}
      \caption{Overview of the \texttt{R} functions and formulas used for fitting the models presented in this study. The overview includes
        the model ID,
        the name of the \texttt{R} package (and its version number) which provided the model fitting function,
        the name of the \texttt{R} model fitting function,
        and the formula used in the model fitting function call.
        In the case of GAMLSS1, GAMLSS2, and GAMLSS3, the formula applies only to the location parameter of the assumed probability distribution.
        For all other distribution parameters, the formula was \texttt{gha \textasciitilde{} 1}. \\
        \texttt{gha}: basal area variable \\
        \StandAgeVariableR{}: stand age variable \\
        \SiteClassVariableR{}: site class variable
        \label{tab:PresentedModelsOverviewFormulas}} \\
      \toprule
      Model ID & Package (Version) & Function & Formula \\
      \midrule
      \endhead
      \bottomrule
      \endlastfoot
      GAM1 & \texttt{mgcv} (1.8.22) & \texttt{gam} & \texttt{gha \textasciitilde{} s(\StandAgeVariableR{}) + s(\SiteClassVariableR{})} \\
      GAM2 & As above & As above & \texttt{gha \textasciitilde{} s(\StandAgeVariableR{}) + \SiteClassVariableR{}} \\
      SCAM1 & \texttt{scam} (1.2.2) & \texttt{scam} & \texttt{gha \textasciitilde{} s(\StandAgeVariableR{}, bs = "micv") + \SiteClassVariableR{}} \\
      GAMLSS1 & \texttt{gamlss} (5.0.4) & \texttt{gamlss} & \texttt{gha \textasciitilde{} ps(\StandAgeVariableR{}) + ps(\SiteClassVariableR{})} \\
      GAMLSS2 & As above & As above & \texttt{gha \textasciitilde{} ps(\StandAgeVariableR{}) + \SiteClassVariableR{}} \\
      GAMLSS3 & As above & As above & \texttt{gha \textasciitilde{} pbm(\StandAgeVariableR{}) + \SiteClassVariableR{}} \\
      \bottomrule
    \end{longtabu}}
\end{table}

One main goal in model formulation was to ensure that the fitted models were capable of separating the effects of stand age and site class on predicted basal area.  However, top height (\(h_{100}\)) as the predictor variable was considered unsuited for achieving this goal, since past experience had shown that using this dimension as the predictor variable does not allow inclusion of a separate site class-effect in the model.  Therefore, 2 new variables were calculated:  a stand age variable and a site class variable, each of which was calculated in such a way as to exclude the effect of the other. The stand age variable was calculated using the equation
\begin{equation}
  \label{eq:StandAgeVariable}
  h_{100}(x)_{\text{I. YC}} = \beta_0 + \beta_1 \cdot \ln(x) + \beta_2 \cdot \ln(x)^2 + SI_{\text{I. YC}} \cdot \bigl(\beta_3 + \beta_4 \cdot \ln(x)\bigr)
\end{equation}
\parencite{Nagel1999}, where \(h_{100}(x)_{\text{I. YC}}\) is the top height in \si{\meter} at age \(x\) if the stand were yield class I, \(SI_{\text{I. YC}}\) is the species-specific top height (\(h_{100}\)) at age \SI{100}{\year} of a stand of yield class I and all other terms have the same meaning as in \RefEq{eq:NagelFunctionSolvedForSI}.  The values for \(SI_{\text{I. YC}}\) were taken from \textcite{Schober1995} and are reported in table \RefTab{tab:SIYieldClassI}.  By setting \(SI_{\text{I. YC}}\) to a species-specific constant, it was possible to exclude any site class-effects from \RefEq{eq:StandAgeVariable}.  Thus, \(h_{100}(x)_{\text{I. YC}}\) only depends on stand age and was therefore chosen as the stand age variable for model fitting.

\begin{table}[H]
  {\tabulinesep=2mm
    \begin{longtabu}{l S L}
      \caption{Species-specific \(SI_{\text{I. YC}}\) values used in \RefEq{eq:StandAgeVariable}.  Values are top height (\(h_{100}\)) at age \SI{100}{\year} of a stand of yield class I as reported in \textcite{Schober1995}
        \label{tab:SIYieldClassI}} \\
      \toprule
      Species & {\(SI_{\text{I. YC}}\) [\si{\meter}]} & \\
      \midrule
      \endhead
      \bottomrule
      \endlastfoot
      Beech & 32.4 \\
      Spruce & 35.1 \\
      \bottomrule
    \end{longtabu}}
\end{table}




\begin{table}[H]
  {\tabulinesep=2mm
    \begin{longtabu}{l l L}
      \caption{Overview of the \texttt{R} distribution functions used for the models presented in this study.  The overview includes
        the model ID,
        the name of the \texttt{R} package (and its version number) which provided the distribution function,
        and the \texttt{R} call of the distribution function.
        \label{tab:PresentedModelsOverviewDistributions}} \\
      \toprule
      Model ID & Package (Version) & Function call \\
      \midrule
      \endhead
      \bottomrule
      \endlastfoot
      GAM1 & \texttt{stats} (3.3.3) & \texttt{Gamma(link = "log")} \\
      GAM2 & As above & As above \\
      SCAM1 & As above & As above \\
      GAMLSS1 & \texttt{gamlss.dist} (5.0.3) & \texttt{BCCGo()} \\
      GAMLSS2 & As above & As above \\
      GAMLSS3 & As above & As above \\
      \bottomrule
    \end{longtabu}}
\end{table}

%%% Local Variables:
%%% mode: latex
%%% TeX-master: "MasArThesis.tex"
%%% End:
