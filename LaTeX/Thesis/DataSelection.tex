\subsection{Data selection}
The present study aims at predicting maximum basal area of even-aged pure \Beech{} and \Spruce{} stands by fitting statistical models to real-world data sets.  This approach necessitates that the data sets used for model fitting only contain observations meeting all of the following requirements:
\begin{enumerate}
\item The observations belong to an even-aged stand.
\item The observations belong to a pure stand of the respective species.
\item The observations belong to a stand with maximum basal area.
\end{enumerate}
The provided data sets were assumed to meet the first requirement.  The second and third requirement were, however, not met by all observations contained in the data sets.  Therefore, a selection of observations had to be made.  With respect to the second requirement, selection was based on crown projection area: an observation was considered a pure stand observation and consequently selected if either \Beech{} or \Spruce{} held \SI{70}{\percent} or more of the stand’s total crown projection area.  The approach for selecting observations which meet the third requirement (termed ``data selection mechanism'') sets species-specific thresholds for the slope of the equation
\begin{equation}
  \label{eq:Reineke}
  \log (N) = s \cdot \log (D) + k ,
\end{equation}
where \(N\) is stand density per unit area, \(s\) is the slope, \(D\) is average diameter by basal area, and \(k\) is a species-specific constant \parencite{Reineke1933}.  It examines observations belonging to the same sample plot in chronological order, while attempting to answer 3 questions:
\begin{enumerate}
\item Is the current observation preceded and/or followed by another observation?
\item If the current observation is preceded by another observation, is the slope of \Cref{eq:Reineke} between the current and the previous observation greater than or equal to the species-specific lower threshold?
\item If the current observation is followed by another observation, is the slope of \Cref{eq:Reineke} between the current and the following observation lower than or equal to the species-specific upper threshold?
\end{enumerate}
An observation is considered to be a maximum basal area observation and consequently selected if and only if all questions are answered positively.  The slope thresholds are reported in \Cref{tab:ReinekeSlopeThresholds} and form a range of allowed slopes.  For \Beech{} it lies between \num{-2.91} and \num{-0.9}, whereas for \Spruce{} it lies between \num{-2.82} and \num{-0.65}.
\Cref{fig:logDlogNPlotsBeforeAfterDataSelection} may aid in understanding how the data selection mechanism works.  It depicts the observed relationship between stand density and average diameter on a double-logarithmic scale before and after application of the data selection mechanism for each species.  In all plots of the figure, the dashed and solid black lines exemplify the species-specific upper and lower slope thresholds, respectively.  Thus, an observation from a plot in column A (observations before application of the data selection mechanims) is present in the corresponding plot of column B (observations after application of the data selection mechanism) if
\begin{itemize}
\item it is connected to a previous and/or following observation by a colored line \\
  and
\item the slope of the line connecting it to the previous observation (if present) is higher than the slope of the solid black line \\
  and
\item the slope of the line connecting it to the following observation (if present) is lower than the slope of the dashed black line.
\end{itemize}

The data sets of selected observations will be further described in the following sections.

\begin{figure}[h]
  \centering
  \includegraphics[width=1.0\textwidth]{../../Graphics/Thesis/logDlogNPlotsBeforeAfterDataSelection.pdf}
  \caption{Observed relationship between stand density \(N\) and average diameter \(D\) in pure stands of \Beech{} (top row) and \Spruce{} (bottom row) on double-logarithmic scale before application of the data selection mechanism (column A) and after application of the data selection mechanism (column B).  Each colored dot represents one observation.  Colored lines connect observations belonging to the same sample plot.  Dashed black lines exemplify the species-specific upper slope threshold used in the data selection mechanism.  Solid black lines exemplify the species-specific lower slope threshold used in the data selection mechanism.  See \Cref{tab:ReinekeSlopeThresholds} for the threshold values.  \\
    % top row: \Beech{} \\
    % bottom row: \Spruce{} \\
    \(n_{\text{obs}}\): number of observations \\
    \(n_{\text{sp}}\): number of sample plots}
  \label{fig:logDlogNPlotsBeforeAfterDataSelection}
\end{figure}

\newpage{}  %% TESTING (this is meant only to prevent "misplaced \cr" errors due to the following longtable stretching across page breaks)
\begin{singlespace}
  {\tabulinesep=2mm
    \begin{longtabu}{L S S}
      \caption{Species-specific lower and upper threshold for the slope \(s\) of \Cref{eq:Reineke} used in the data selection mechanism.  \label{tab:ReinekeSlopeThresholds}} \\
      \toprule
      % Species & {Lower threshold} & {Upper threshold} \\
      & {Beech} & {Spruce} \\
      \midrule
      \endhead
      \bottomrule
      \endlastfoot
      % \Beech{} & -2.91 & -0.9 \\
      % \Spruce{} & -2.82 & -0.65 \\
      Lower threshold & -2.91 & -2.82 \\
      Upper threshold & -0.9 & -0.65 \\
    \end{longtabu}
  }
\end{singlespace}

\clearpage{}

%%% Local Variables:
%%% mode: latex
%%% TeX-master: "MasArThesis.tex"
%%% End:
