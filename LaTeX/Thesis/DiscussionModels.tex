\subsection{Models}

The logarithm was chosen as the link function for the location parameter in all models in order to prevent unrealistic predictions of negative basal area.

The confidence bands of the estimated smooth function effect in all models contained non-zero estimates (cp. \Cref{fig:GAM1EffectStandAgeVariable,fig:GAM1EffectProductivityIndexVariable,fig:GAM2EffectStandAgeVariable,fig:SCAM1EffectStandAgeVariable,fig:GAMLSS1EffectStandAgeVariable,fig:GAMLSS1EffectProductivityIndexVariable,fig:GAMLSS2EffectStandAgeVariable,fig:GAMLSS3EffectStandAgeVariable}).  This suggests that the stand age variable in all models and the \ProductivityIndexVariableText{} in models GAM1 and GAMLSS1 is related to the response variable \parencite{Wood2001}.

Using the stand age variable, rather than stand age directly, as a predictor variable reduces comparability of models across species, since its calculation uses species-specific coefficients.

The sequence of models presented here, from GAM1 over GAM2 to SCAM1 and from GAMLSS1 over GAMLSS2 to GAMLSS3, may be seen as a sequence of increasingly constrained models.  GAM1 and GAMLSS1 contain the least constraints:  the effect of stand age variable and \ProductivityIndexVariableText{} on basal area is completely unrestricted and allowed to take on any shape.  In models GAM2 and GAMLSS2, the \ProductivityIndexVariableText{} is constrained to have a linear, rather than a smooth effect on basal area.  Finally, in models SCAM1 and GAMLSS3 the stand age variable smooth function is also constrained, namely to be either increasing and concave or monontone increasing, respectively.  The fact that model GAMLSS3 only imposes a monotone increasing constraint, rather than an increasing and concave constraint like model SCAM1, is due to the fact that no such constraint has been implemented in \texttt{R} package \texttt{gamlss} as of version \gamlssPackageVersion{}.
This imposition of increasingly strict constraints was motivated by the implausible results of the lesser restricted models.
In model GAM1, yield class \num{1} in \Beech{} performs better than yield class \num{0} while
yield class \num{-2} in \Spruce{} performs worse than yield class \num{1} throughout all stand ages between \SIrange{2}{160}{\year} (cp. \Cref{fig:GAM1StandAgeBasalAreaObservationsPredictionsYieldClassClassification}).
In model GAMLSS1, the same problem regarding predictions for \Beech{} persists (cp. \Cref{fig:GAMLSS1StandAgeBasalAreaObservationsPredictionsYieldClassClassification}).  Additionally, in both models the distance between yield classes \numlist{3;2} in \Beech{} and \numlist{-1;0} in \Spruce{} is much larger than between other adjacent yield classes.  Compared to the values reported in \textcite{Schober1995}, which are depicted in \Cref{fig:StandAgeBasalAreaSchober}, this irregularity in distance appears implausible.
In models GAM2 and GAMLSS2, linearization of the \ProductivityIndexVariableText{} eliminated these irregularities (cp. \Cref{fig:GAM2StandAgeBasalAreaObservationsPredictionsYieldClassClassification,fig:GAMLSS2StandAgeBasalAreaObservationsPredictionsYieldClassClassification}).  However, basal area predictions remain implausible, since in the case of \Beech{}, basal area is predicted to decrease after approximately \SI{80}{\year} in all yield classes, whereas in the case of \Spruce{}, basal area increases indefinitely in all yield classes.  Both results are not in conjunction with values reported in \textcite{Schober1995} (moderate thinning), which suggest no decrease in basal area for \Beech{} at least up to \SI{150}{\year} for yield classes \numrange{1}{4}, and a decrease in basal area for yield classes \numrange{1}{4} of \Spruce{} after \SI{100}{\year} the latest (cp. \Cref{fig:StandAgeBasalAreaSchober}).  Similarly, \textcite{Franz1965} reports \Spruce{} stands to reach a maximum basal area at around \SI{120}{\year}.
These remaining implausibilities were partly met by the additional constraints imposed in models SCAM1 and GAMLSS3 (cp. \Cref{fig:SCAM1StandAgeBasalAreaObservationsPredictionsYieldClassClassification,fig:GAMLSS3StandAgeBasalAreaObservationsPredictionsYieldClassClassification}).  Basal area predictions for \Beech{} show no decrease for stand ages up to \SI{160}{\year}, but rather an asymptotic behavior, reaching their maximum at around \SI{67}{\year} in the case of SCAM1 and around \SI{80}{\year} in the case of GAMLSS3.  While neither of these values is in accordance with the basal area development reported by \textcite{Schober1995}, the predictions for \Beech{} of models SCAM1 and GAMLSS3 nevertheless appear more plausible than those of models GAM2 and GAMLSS2, respectively.  In the case of \Spruce{}, the difference between model SCAM1, which imposes both an increasing as well as a concavity constraint, and model GAMLSS3, which only imposes a monontone increasing constraint, becomes apparent.  While in both models basal area predictions increase indefinitely, the increase is remarkably smaller in SCAM1, which predicts, at \SI{160}{\year}, a basal area of roughly \SI{93}{\square\meter\per\hectare} for yield class \num{-2} and of about \SI{55}{\square\meter\per\hectare} for yield class \num{4}, whereas GAMLSS3, for the same age and yield classes, predicts a basal area of approximately \SIlist{138;81}{\square\meter\per\hectare}, respectively.
In \Cref{fig:SCAM1TopHeightBasalAreaObservationsPredictionsYieldClassClassification}, which depicts predicted basal area over top height for model SCAM1, in both species the order of prediction curves initially is the inverse of yield class order, but shifts during a certain range of top heights, so that finally it is the same as yield class order.
The same is true for \Beech{} in model GAMLSS3 (cp. \Cref{fig:GAMLSS3TopHeightBasalAreaObservationsPredictionsYieldClassClassification}).  These shifts are due to the fact that prediction curves of low productivity yield classes level off or show a decrease in slope at lower top heights than do curves of higher productivity yield classes.
The reason for this becomes apparent when considering \Cref{fig:StandAgeTopHeightTestData}, which depicts the relationship between top height and stand age in the test data sets (on which model predictions were based):  for a given top height, age increases as yield class worsens.  Thus, the asymptotic behavior or the decrease in slope of the prediction curves occurs at lower top heights the worse a stand’s yield class is, but still at approximately the same stand age in all yield classes for each species (cp. \Cref{fig:SCAM1StandAgeBasalAreaObservationsPredictionsYieldClassClassification,fig:GAMLSS3StandAgeBasalAreaObservationsPredictionsYieldClassClassification}).  Similar shifts of curve order can be observed in the predictions of other models as well, owing to the same properties of the test data sets.
All things considered, the chosen constraints improved plausibility of model predictions.

However, the more constrained models also generally achieved higher (i.e., worse) AIC scores than the less constrained models (cp. \Cref{tab:AICOverview}), with the notable exception of GAMLSS3 for \Spruce{}, which achieved the lowest AIC for that species.  A reason for this discrepancy between plausibility and statistical power of the models may lie in the incompleteness of the training data sets discussed above.  At the same time, the GAMLSSs generally achieved a lower AIC than the correpsonding GAM or SCAM, with the exception of GAMLSS1 in \Spruce{} which achieved a slightly higher score than GAM1.  This suggests that the increased uncertainty of GAMLSSs compared to GAMs or SCAMs was outweighed by their increased flexibility.

Since the training data sets only consisted of pure even-aged stands, model predictions are also only valid for such stands.  In order to acquire valid maximum basal area predictions for mixed and/or uneven-aged stands, the models would need to be fitted using appropriate training data sets.

Since GAMLSSs directly estimate all parameters of the assumed probability distribution, they also offer the option of predicting specific distribution quantiles \parencite{Yee2004}.  As shown by \textcite{Woerdehoff2014}, this is particularly useful in the context of maximum basal area modeling insofar as models may be fitted using non-maximum basal area observations, while predicting upper quantiles to obtain function envelopes, which may be considered a close approximation of true maximum basal area predictions.  Since the training data sets in this study here were assumed to contain only maximum basal area observations after application of the data selection mechanism, this option was not examined.  It nevertheless does warrant further research.

\begin{figure}[h]
  \centering
  \includegraphics[width=1.0\textwidth]{../../Graphics/Thesis/StandAgeBasalAreaSchober.pdf}
  \caption{Basal area (\(G\)) over stand age of \Beech{} (top) and \Spruce{} (bottom) as reported in \textcite{Schober1995} (moderate thinning).  Color signifies yield class.}
  \label{fig:StandAgeBasalAreaSchober}
\end{figure}

\begin{figure}[h]
  \centering
  \includegraphics[width=1.0\textwidth]{../../Graphics/Thesis/StandAgeTopHeightTestData.pdf}
  \caption{Top height (\(\TopHeightMath{}\)) over stand age of \Beech{} (top) and \Spruce{} (bottom) in the test data.  Test data was generated using \Cref{eq:NagelFunctionSolvedForTopHeight}.  Line color has the same meaning as in \Cref{fig:GAMLSS3TopHeightBasalAreaObservationsPredictionsYieldClassClassification}, namely:  Color signifies the (fractional) yield class of the test data, ranging from red (worst yield class observed in the training data) over yellow and green to blue (best yield class observed in the training data).  Yield class classification was based on \ProductivityIndexText{} as given by \Cref{eq:NagelFunctionSolvedForProductivityIndex} (rounded to one decimal digit), using \Cref{tab:SchoberProductivityIndices} as reference.  Note the different yield class ranges in both plots.}
  \label{fig:StandAgeTopHeightTestData}
\end{figure}

\clearpage{}

%%% Local Variables:
%%% mode: latex
%%% TeX-master: "MasArThesis.tex"
%%% End:
