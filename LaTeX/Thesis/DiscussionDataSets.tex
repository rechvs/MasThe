\subsection{Data sets}

Although sample plots in both species cover a large range of geographical locations and altitudes (cp. \Cref{fig:LocationsSamplePlots,fig:SpeciesAltitudeOfSamplePlots}), neither of these properties could be harnessed as a predictor variable in either species, apparently due to the small data set sizes.

Considering \Cref{fig:StandAgeTopHeightYieldClassClassification}, the method of yield class classification yielded plausible results in both data sets:  for a given top height, yield class gradually worsens as stand age increases.  Both \Cref{eq:NagelFunctionSolvedForProductivityIndex}, which was used for yield class classification of observed top heights, as well as \Cref{eq:NagelFunctionSolvedForTopHeight}, which was used for test data generation, are based on the function developed by \textcite{Nagel1999}.  This ensured comparable yield class classification of training as well as test data.

The training data sets only partially cover the range of site characteristics encountered in forest stands in northwestern Germany.
This fragmentariness of the training data sets reduces generalizability of model predictions insofar as predictions of stands that are poorly represented in the training data sets cannot be relied upon in the same manner as those for stands which are well mirrored in them can be.
In the case of \Beech{}, stand ages below \SI{30}{\year} are completely absent, while in the case of \Spruce{}, only \num{2} observations of stand ages above \SI{100}{\year} are present (cp. \Cref{fig:StandAgeTopHeightYieldClassClassification}).
Top heights below \SI{15}{\meter} are completely absent from the \Beech{} data set and only represented by \num{2} observations in the \Spruce{} data set (cp. \Cref{fig:StandAgeTopHeightYieldClassClassification}).
With respect to productivity, extreme sites are not as well represented in the training data sets as intermediate sites (cp. \Cref{fig:StandAgeProductivityIndexYieldClassClassification}), with sites below yield class \num{3} being completely absent from the \Beech{} data set.  At the same time there is a strong negative correlation between stand age and \ProductivityIndexText{} in both data sets, meaning that young lowly productive stands as well as old highly productive stands are particularly underrepresented in the training data sets.

%%% Local Variables:
%%% mode: latex
%%% TeX-master: "MasArThesis.tex"
%%% End:
