\subsection{Test data}

As can be seen in \Cref{fig:StandAgeTopHeightTestData}, \Cref{eq:NagelFunctionSolvedForTopHeight} yields negative top heights for low stand ages (up to \SI{25}{\year} for yield class \num{3} and better of \Beech{} and up to \SI{21}{\year} for yield class \num{4} and better of \Spruce{}).  This affects model results insofar, as in both species basal area predictions for a top height of \SI{0}{\meter} exceed \SI{0}{\square\meter\per\hectare}, with the magnitude of excess depending on yield class (cp. \Cref{fig:GAM1TopHeightBasalAreaObservationsPredictionsYieldClassClassification,fig:GAM2TopHeightBasalAreaObservationsPredictionsYieldClassClassification,fig:SCAM1TopHeightBasalAreaObservationsPredictionsYieldClassClassification,fig:GAMLSS1TopHeightBasalAreaObservationsPredictionsYieldClassClassification,fig:GAMLSS2TopHeightBasalAreaObservationsPredictionsYieldClassClassification,fig:GAMLSS3TopHeightBasalAreaObservationsPredictionsYieldClassClassification}).  Thus, model predictions for low stand ages are not as dependable as those for higher stand ages.

%%% Local Variables:
%%% mode: latex
%%% TeX-master: "MasArThesis.tex"
%%% End:
