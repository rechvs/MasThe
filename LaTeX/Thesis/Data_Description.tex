\subsection{Description of data set}

The total number of sample plots was \num{18} for \beech{} and \num{28} for \spruce{}.  The geographical location of the sample plots is reported in \reffig{fig:Locations_Sample_Plots}.  The dots in both plots of the figure do not add up to the total number of sample plots of the respective species because some sample plots were part of the same trial and therefore shared the geographical location.  As can be seen from \reffig{fig:Altitude_Sample_Plots}, sample plots of both species cover a wide range of altitudes above sea level (\beech{}: \SIrange{40}{565}{\meter}, \spruce{}: \SIrange{20}{750}{\meter}).

% CONTINUE HERE with describing yield classes of the sample plots.
  
\begin{figure}[H]
  \centering
  \includegraphics[width=1.0\textwidth]{../../Graphics/Thesis/Locations_Sample_Plots.pdf}
  \caption{Geographical location of the trial plots of \beech{} (left) and \spruce{} (right) in Germany.}
  \label{fig:Locations_Sample_Plots}
\end{figure}

\begin{figure}[H]
  \centering
  \includegraphics[width=0.5\textwidth]{../../Graphics/Thesis/Altitude_Sample_Plots.pdf}
  \caption{Altitude above sea level of sample plots.}
  \label{fig:Altitude_Sample_Plots}
\end{figure}

% \newpage{}  %% TESTING (this is meant only to prevent "misplaced \cr" errors due to the following longtable stretching across page breaks)
% \begin{singlespace}
  % {\tabulinesep=2mm
    % \begin{longtabu}{L S S}
      % \caption{Summary statistics of the altitude above sea level of the sample plots. \label{tab:AltitudeSummaries}} \\
      % \toprule
      % Statistic & {Beech} & {Spruce} \\
      % \midrule
      % \endhead
      % \bottomrule
      % \endlastfoot
      % Minimum & 40 & 20 \\
      % Mean & 315.6 & 404 \\
      % Median & 363.5 & 445 \\
      % Maximum & 565 & 750 \\
    % \end{longtabu}
  % }
% \end{singlespace}

%%% Local Variables:
%%% mode: latex
%%% TeX-master: "MasAr_Thesis.tex"
%%% End:
