\section{Selection of data}

The present study aims at predicting maximum basal area by fitting statistical models to real-world data.  This approach requires that the observations selected as the training data are (close to) maximum basal area observations.  Thus, we need to provide an answer to the following question: ``Which basal area observations from a large pool of observations can be considered \emph{maximum} basal area observations?''

\subsection{Maximum basal area and self-thinning}

In his definition of ``maximum basal area'', \textcite{Assmann1970} points out that maximum basal area is only achieved in stands which have not actively been thinned.  In this context, ``active thinning'' means a reduction of stand density exceeding natural density dependent mortality, also known as ``natural thinning'' \parencite{SAF1958} or ``self-thinning'' \parencite{Roehrig1992}.  Self-thinning, by definition, is a naturally occurring phenomenon.  However, the cause of thinning, be it natural inter-plant competition or human influence, is of no relevance to the present study since the end result is the same: a reduction of stand density.  What is relevant is the actual extent or rate of thinning, since a thinning rate below or above the self-thinning rate will lead to the basal area being above or below its natural maximum.  Thus, data selection was based on the following rationale: a stand is considered to have maximum basal area, as long as its thinning rate is roughly equal to the self-thinning rate of the respective species.  This, however, requires knowledge of the self-thinning modalities of forest stands in general and the species in question in particular.

\textcite{Reineke1933} proposed that every forest stand, regardless of tree species and stand development, is subject to the same self-thinning rate given by the equation
\begin{equation}
  \label{eq:reineke}
  \log (N) = s \cdot \log (D) + k ~,
\end{equation}
with \(s = -1.605\) being the species-independent self-thinning rate and \(k\) being a species-specific constant.  According to \textcite{Reineke1933}, every increase in the logarithm of mean diameter leads to a \num{1.605}-fold decrease in the logarithm of stand density, regardless of species.

However, the universal applicability of \refeq{eq:reineke} has been called into question.  As an alternative approach, \textcite{Charru2012} proposed the use of a quadratic, rather than a linear logarithmic diameter term in the right hand side of \refeq{eq:reineke}. A similar approach is taken by \textcite{Schuetz2008,Schuetz2010,Zeide1995}, whose results suggested to add a quadratic logarithmic diameter term to the right hand side of \refeq{eq:reineke}.  \textcite{Meyer1938} found that for \ponderosa{}, the line reported by \textcite{Reineke1933} had to be changed to a slightly concave curve in order to fit observations.  

Despite the apparent shortcomings of \refeq{eq:reineke}, its general form has been upheld by other studies.  Building on the findings of \textcite{Drew1979}, \textcite{VanderSchaaf2010,VanderSchaaf2008} argued that \refeq{eq:reineke} is valid, but only during a specific phase of stand development.  Similarly, \textcite{Zeide1985} suggested that \refeq{eq:reineke} is only applicable during the stage of full canopy closure.  A different approach encountered in the literature is to use a species-specific slope rather than the constant of \num{-1.605} reported by \textcite{Reineke1933} \parencite{MacKinney1935,Pretzsch2005,Charru2012,Pretzsch2006,Río2001,Sterba1987,Vacchiano2013,Vospernik2015,Zeide1985,Zeide1987,VanderSchaaf2007}.  \reftab{tab:SpeciesSpecificReinekeSlopes} lists several example slopes for \beech{} and \spruce{} as reported in the literature.  \textcite{Pretzsch2000,Pretzsch2002} showed that the rule of \textcite{Reineke1933} may be considered a special case of the \num{-3 / 2} power rule of \textcite{Yoda1963}, claiming that as long as the \num{-3 / 2} power rule is valid, species-specific deviations from Reineke’s constant of \num{-1.605} are a result of species-specific diameter-biomass relationships.

\subsection{Mechanism for selecting maximum basal area observations}

The approach for selecting maximum basal area observations employed in this study uses \refeq{eq:reineke} in combination with species-specific self-thinning rates.  It examines observations belonging to a stand in chronological order, while attempting to answer 3 questions:
\begin{enumerate}
\item Is the current observation preceded and/or followed by another observation?
\item If the current observation is preceded by another observation, is the slope \(s\) of \refeq{eq:reineke} between the current and the previous observation greater than or equal to the species-specific lower threshold?
\item If the current observation is followd by another observation, is the slope \(s\) of \refeq{eq:reineke} between the current and the following observation lower than or equal to the species-specific upper threshold?
\end{enumerate}
An observation is considered to be a maximum basal area observation if and only if all questions are answered positively.
The slope thresholds are reported in \reftab{tab:ReinekeSlopeThresholds} and form a range of allowed thinning rates.  While being based on literature values, they are essentially arbitrary and a compromise of 2 conflicting goals: select enough observations to ensure generalizable models but do not select observations which clearly do not show self-thinning.
% Figure \ref{fig:DataSelectionFlowChart} provides a flowchart of the mechanism.

% \newpage{}
% \begin{landscape}
  \begin{figure}[h]
    \caption{Flowchart representing the mechanism used for identifying and removing observations which were not considered to be maximum stand density-observations based on the slope \(s\) of \RefEq{eq:Reineke} between adjacent observations. \\
      \(s_l\): species-specific lower threshold for the slope of \RefEq{eq:Reineke} (cp. \RefTab{tab:ReinekeSlopeThresholds}). \\  %% Note: the hyperlinks in this caption are not placed ontop of the corresponding printed character, but at a seemingly random distance behind it. I am currently (2017-10-03) unsure about the cause of this and thus don’t know of any fix for it.
      \(s_f\): slope of \RefEq{eq:Reineke} between current and following observation. \\
      \(s_p\): slope of \RefEq{eq:Reineke} between current and previous observation. \\
      \(s_u\): species-specific upper threshold for the slope of \RefEq{eq:Reineke} (cp. \RefTab{tab:ReinekeSlopeThresholds}).}
    \label{fig:DataSelectionFlowChart}
  {
    \footnotesize
    % \small
    \begin{tikzpicture}[>=stealth,node distance = 0.5]
      %% Start.
      \node (Start1-1) at(0, 0) [DataSelectionFlowChartStart] {Start loop at first \\ observation.};
      %% Questions.
      \node (Question1-1) at(7.5, 0) [DataSelectionFlowChartQuestion] {At which observation \\ are we~?};
      \node (Question2-1) at(5, -3.5) [DataSelectionFlowChartQuestion] {Is \(s_f \leq s_u\)~?};
      \node (Question2-2) at(10, -3.5) [DataSelectionFlowChartQuestion] {Is \(s_l \leq s_p\) \\ AND \(s_f \leq s_u\)~?};
      \node (Question2-3) at(15, -3.5) [DataSelectionFlowChartQuestion] {Is \(s_l \leq s_p\)~?};
      %% Actions.
      \node (Action1-1) at(0,-7) [DataSelectionFlowChartAction] {delete \\ observation};
      \node (Action1-2) at(6,-7) [DataSelectionFlowChartAction] {delete \\ observation};
      \node (Action1-3) at(11,-7) [DataSelectionFlowChartAction] {delete \\ observation};
      \node (Action1-4) at(16,-7) [DataSelectionFlowChartAction] {delete \\ observation};
      \node (Action2-1) at(7.5, -9) [DataSelectionFlowChartAction] {move to \\ following observation};
      %% Results (drawn last because they should cover any lines with which they overlap).
      \node (Result1-1) at(0, -1.9) [DataSelectionFlowChartResult] {first, which is also the last};
      \node (Result1-2) at(5, -1.9) [DataSelectionFlowChartResult] {first, but not last};
      \node (Result1-3) at(10, -1.9) [DataSelectionFlowChartResult] {intermediate};
      \node (Result1-4) at(15, -1.9) [DataSelectionFlowChartResult] {last, but not first};
      \node (Result2-1) at(3.5, -5.5) [DataSelectionFlowChartResult] {TRUE};
      \node (Result2-2) at(6, -5.5) [DataSelectionFlowChartResult] {FALSE};
      \node (Result2-3) at(8.5, -5.5) [DataSelectionFlowChartResult] {TRUE};
      \node (Result2-4) at(11, -5.5) [DataSelectionFlowChartResult] {FALSE};
      \node (Result2-5) at(13.5, -5.5) [DataSelectionFlowChartResult] {TRUE};
      \node (Result2-6) at(16, -5.5) [DataSelectionFlowChartResult] {FALSE};
      %% End.
      \node (End1-1) at(0, -9) [DataSelectionFlowChartEnd] {Exit loop.};
      %% Horizontal lines and arrows.
      \draw [DataSelectionFlowChartArrow] (Start1-1.east) -- (Question1-1.west); \draw [DataSelectionFlowChartArrow] (19, 0) -- (Question1-1.east);
      \draw [DataSelectionFlowChartLine] (3.5, -8) -- (11, -8); \draw [DataSelectionFlowChartLine] (13.5, -8) -- (16, -8);
      \draw [DataSelectionFlowChartLine] (Action2-1.east) -- (19, -9);
      \draw [DataSelectionFlowChartLine] (13.5, -10.25) -- (0, -10.25);
      %% Vertical lines.
      \draw [DataSelectionFlowChartArrow] (Result1-1.south) -- (Action1-1.north); \draw [DataSelectionFlowChartArrow] (Action1-1.south) -- (End1-1.north); \draw [DataSelectionFlowChartArrow] (0, -10.25) -- (End1-1.south); 
      \draw [DataSelectionFlowChartArrow] (Result1-2.south) -- (Question2-1.north);
      \draw [DataSelectionFlowChartArrow] (Result1-3.south) -- (Question2-2.north);

      \draw [DataSelectionFlowChartArrow] (7.5, -8) -- (Action2-1.north);

      \draw [DataSelectionFlowChartArrow] (Result1-4.south) -- (Question2-3.north);
      
      \draw [DataSelectionFlowChartLine] (Result2-1.south) -- (3.5, -8);
      \draw [DataSelectionFlowChartArrow] (Result2-2.south) -- (Action1-2.north); \draw [DataSelectionFlowChartLine] (Action1-2.south) -- (6, -8);

      \draw [DataSelectionFlowChartLine] (Result2-3.south) -- (8.5, -8);
      \draw [DataSelectionFlowChartArrow] (Result2-4.south) -- (Action1-3.north); \draw [DataSelectionFlowChartLine] (Action1-3.south) -- (11, -8);

      \draw [DataSelectionFlowChartLine] (Result2-5.south) -- (13.5, -8.9); \draw [thick] (13.5, -8.9) arc [start angle = 90, end angle = 270, radius = 0.1]; \draw [DataSelectionFlowChartLine] (13.5, -9.1) -- (13.5, -10.25);

      \draw [DataSelectionFlowChartArrow] (Result2-6.south) -- (Action1-4.north); \draw [DataSelectionFlowChartLine] (Action1-4.south) -- (16, -8);
      
      \draw [DataSelectionFlowChartLine] (19, -9) -- (19, 0);
      %% Slanted lines.
      \draw [DataSelectionFlowChartLine] (Question1-1.south) -- (Result1-1.north);
      \draw [DataSelectionFlowChartLine] (Question1-1.south) -- (Result1-2.north);
      \draw [DataSelectionFlowChartLine] (Question1-1.south) -- (Result1-3.north);
      \draw [DataSelectionFlowChartLine] (Question1-1.south) -- (Result1-4.north);

      \draw [DataSelectionFlowChartLine] (Question2-1.south) -- (Result2-1.north);
      \draw [DataSelectionFlowChartLine] (Question2-1.south) -- (Result2-2.north);
      \draw [DataSelectionFlowChartLine] (Question2-2.south) -- (Result2-3.north);
      \draw [DataSelectionFlowChartLine] (Question2-2.south) -- (Result2-4.north);
      \draw [DataSelectionFlowChartLine] (Question2-3.south) -- (Result2-5.north);
      \draw [DataSelectionFlowChartLine] (Question2-3.south) -- (Result2-6.north);
      %% Legend.
      % \draw [thick,dashed] (-2, -11) rectangle (14, -13);

      % \node[anchor=west] (LegendTitle) at(-1.9, -12) {\textbf{Legend:}};

      % \node (LegendStart)  [DataSelectionFlowChartStart,right=of LegendTitle] {Loop start};

      % \node (LegendQuestion) [DataSelectionFlowChartQuestion,right=of LegendStart] {Question/Test};

      % \node (LegendResult) [DataSelectionFlowChartResult,right=of LegendQuestion] {Answer/Result};

      % \node (LegendAction) [DataSelectionFlowChartAction,right=of LegendResult] {Action};

      % \node (LegendEnd) [DataSelectionFlowChartEnd,right=of LegendAction] {Loop end};
      
    \end{tikzpicture}}
\end{figure}
\end{landscape}

%%% Local Variables:
%%% mode: latex
%%% TeX-master: "MasAr_Thesis.tex"
%%% End:
  %% [2017-10-04: I opted against the figure (and deactivated all references to it) since it does not provide any additional information to the 3-questions list above.]

\newpage{}  %% TESTING (this is meant only to prevent "misplaced \cr" errors due to the following longtable stretching across page breaks)
\begin{singlespace}
  {\tabulinesep=2mm
    \begin{longtabu}{L l l}
      % \caption{Species-specific lower (\(s_l\)) and upper (\(s_u\)) threshold for the slope of \refeq{eq:reineke} used for selecting maximum basal area observations (cp. \reffig{fig:DataSelectionFlowChart}). 
      \caption{Species-specific lower and upper threshold for the slope of \refeq{eq:reineke} used for selecting maximum basal area observations.  \label{tab:ReinekeSlopeThresholds}} \\
      \toprule
      % Species & \(s_l\) & \(s_u\) \\
      Species & Lower threshold & Upper threshold \\
      \midrule
      \endhead
      \bottomrule
      \endlastfoot
      \beech{} & \num{-2.91} & \num{-0.9} \\
      \spruce{} & \num{-2.82} & \num{-0.65} \\
    \end{longtabu}
  }
\end{singlespace}

\newpage{}  %% TESTING (this is meant only to prevent "misplaced \cr" errors due to the following longtable stretching across page breaks)
\begin{singlespace}
  {\tabulinesep=2mm
    \begin{longtabu}{L r r}
      \caption{Species-specific values for parameter \(s\) of \refeq{eq:reineke} for \beech{} and \spruce{} as reported in the literature.  \label{tab:SpeciesSpecificReinekeSlopes}} \\
      \toprule
      Source & beech & spruce \\
      \midrule
      \endfirsthead
      \caption{(continued)} \\
      % Source & beech & spruce \\
      % \midrule
      \endhead
      \bottomrule
      \endlastfoot
      \textcite{Charru2012} & \num{-1.941} & \num{-1.878} \\
      \textcite{Pretzsch2006} & \numrange{-1.873}{-1.723} & \numrange{-1.669}{-1.607} \\
      \textcite{Pretzsch2005} & \num{-1.789} & \num{-1.664} \\
      \textcite{Sterba1987} & & \num{-1.737} \\
      \textcite{Vacchiano2013} & & \num{-1.497} \\
      \textcite{Vospernik2015} & \num{-1.941} & \num{-1.753} \\
    \end{longtabu}
  }
\end{singlespace}

%%% Local Variables:
%%% mode: latex
%%% TeX-master: "MasAr_Thesis.tex"
%%% End:
