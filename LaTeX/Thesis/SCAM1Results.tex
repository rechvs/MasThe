\subsection{SCAM1}

\Cref{fig:SCAM1EffectStandAgeVariable} shows the estimated effect of the stand age variable smooth function over the stand age variable in model SCAM1.  The smooth function used P-splines constrained to be increasing and concave as its basis.  In the case of \Beech{}, the smooth function behaves asymptotic:  for stand age variable values up to circa \SI{25}{\meter} (which corresponds to an age of about \SI{67}{\year}), the smooth function value increases before levelling off.  The \edf{} of the smooth function is \num{1.86}.  In the case of \Spruce{}, the overall curve shape is similar to the corresponding figure of model GAM2 (cp. \Cref{fig:GAM2EffectStandAgeVariable}):  the smooth function value increases steeply for stand age variable values up to \SI{12.5}{\meter} (which corresponds to an age of about \SI{25}{\year}), after which the increase continues, but at a lower rate.  The \edf{} of the smooth function is \num{3.01}.

\Cref{fig:SCAM1EffectProductivityIndexVariable} depicts the estimated linear effect of the \ProductivityIndexVariableText{} over the \ProductivityIndexVariableText{} in model SCAM1.  As in model GAM2, the effect is monotone increasing in both species, having its root at approximately \SI{0}{\meter}.

\Cref{fig:SCAM1QQPlot} shows the quantile-quantile plots for model SCAM1.  In \Beech{}, several residuals of low, intermediate, and high quantiles lie outside the \SI{90}{\percent} reference band.  In \Spruce{}, multiple residuals of low and intermediate quantiles lie outside the reference band.

\Cref{fig:SCAM1FittedValuesResiduals} shows response residuals over fitted values of model SCAM1.  In \Beech{}, residual variance appears to be constant over the range of fitted values depicted, whereas in \Spruce{}, residual variance appears to be higher for fitted values around \num{50} and lower for other values.

\Cref{fig:SCAM1StandAgeBasalAreaObservationsPredictionsYieldClassClassification} depicts basal area over stand age, both observed values as well as predictions of model SCAM1.  In both species, the overall shape of prediction curves follows closely that of the stand age variable smooth function as shown in \Cref{fig:SCAM1EffectStandAgeVariable}:  in \Beech{}, predicted basal area increases up to an age of about \SI{67}{\year} at which it levels off;  in \Spruce{}, predicted basal area increases steeply up to an age of about \SI{25}{\year}, after which the increase continues, albeit at a lower rate.  In both species, curves for different yield classes are clearly separated from each other, with curve order being the inverse of yield class order in both species.

\Cref{fig:SCAM1TopHeightBasalAreaObservationsPredictionsYieldClassClassification} shows basal area over top height, both observed values as well as predictions of model SCAM1.  In \Beech{}, stratification of prediction curves is rather weakly pronounced up to a top height of about \SI{17}{\meter}, with curve order being the inverse of yield class order, i.e., worse yield classes outperforming better yield classes.  Between top heights of \SIrange{17}{31}{\meter}, curve order switches due to basal area of worse yield classes levelling off at lower top heights than that of better yield classes, so that for top heights above \SI{31}{\meter} curve order follows yield class order.  In \Spruce{}, a similar pattern is visible:  for top heights up to \SI{9}{\meter}, curve order is the inverse of yield class order;  between a top height of \SIrange{9}{26}{\meter}, curve order switches due to curve slopes of worse yield classes decreasing at lower top heights than that of better yield classes, so that for top heights above \SI{26}{\meter} curve order follows yield class order.

\begin{figure}[h]
  \centering
  \includegraphics[width=1.0\textwidth]{../../Graphics/Thesis/SCAM1/SCAM1EffectStandAgeVariable.pdf}
  \caption{Estimated effect of the stand age variable smooth function (\(s(\StandAgeVariableMath{}, \ldots)\)) over stand age variable (\(\StandAgeVariableMath{}\)) in model SCAM1 for \Beech{} (top) and \Spruce{} (bottom).   Solid lines, dashed lines, vertical bars, and numbers in the y-axis titles have the same meaning as in \Cref{fig:GAM2EffectProductivityIndexVariable}, namely:  Solid lines mark estimates.  Dashed lines mark confidence bands of 2 standard errors width.  Vertical bars mark observed values.  Numbers in the y-axis titles are the effective degrees of freedom of the smooth function.}
  \label{fig:SCAM1EffectStandAgeVariable}
\end{figure}

\begin{figure}[h]
  \centering
  \includegraphics[width=1.0\textwidth]{../../Graphics/Thesis/SCAM1/SCAM1EffectProductivityIndexVariable.pdf}
  \caption{Estimated effect of the parametric productivity index variable term (Partial for \(\ProductivityIndexVariableMath{}\)) over \ProductivityIndexVariableText{} (\(\ProductivityIndexVariableMath{}\)) in model SCAM1 for \Beech{} (top) and \Spruce{} (bottom).   Solid lines, dashed lines, and vertical bars have the same meaning as in \Cref{fig:SCAM1EffectStandAgeVariable}, namely:  Solid lines mark estimates.  Dashed lines mark confidence bands of 2 standard errors width.  Vertical bars mark observed values.  Note the different scaling of the x-axis in both plots.}
  \label{fig:SCAM1EffectProductivityIndexVariable}
\end{figure}

\begin{figure}[h]
  \centering
  \includegraphics[width=1.0\textwidth]{../../Graphics/Thesis/SCAM1/SCAM1QQPlot.pdf}
  \caption{Quantile-quantile plot of residuals of model SCAM1 for \Beech{} (top) and \Spruce{} (bottom).  Black lines, black dots, and gray shaded areas have the same meaning as in \Cref{fig:GAM2QQPlot}, namely:  Black lines are reference lines.  Black dots represent residuals.  Gray shaded areas mark reference bands between the \num{0.05} and \num{0.95} quantiles of predictions (\SI{90}{\percent} level).  Theoretical quantiles and reference bands are based on repeated predictions (\(N = \num{e4}\)) \parencite{Augustin2012}.  Note the different axis scaling in both plots.}
  \label{fig:SCAM1QQPlot}
\end{figure}

\begin{figure}[h]
  \centering
  \includegraphics[width=1.0\textwidth]{../../Graphics/Thesis/SCAM1/SCAM1FittedValuesResiduals.pdf}
  \caption{Observed values minus fitted values (Response residuals) over fitted values of model SCAM1 for \Beech{} (top) and \Spruce{} (bottom).}
  \label{fig:SCAM1FittedValuesResiduals}
\end{figure}

\begin{figure}[h]
  \centering
  \includegraphics[width=1.0\textwidth]{../../Graphics/Thesis/SCAM1/SCAM1StandAgeBasalAreaObservationsPredictionsYieldClassClassification.pdf}
  \caption{Basal area (\(G\)) over stand age of \Beech{} (top) and \Spruce{} (bottom).  Colored lines represent predictions of model SCAM1.  Dots, black lines, and colors have the same meaning as in \Cref{fig:GAM2TopHeightBasalAreaObservationsPredictionsYieldClassClassification}, namely:  Each dot represents one observation.  Black lines connect observations belonging to the same sample plot.  Color signifies the (fractional) yield class of the respective observation or prediction, ranging from red (worst yield class observed) over yellow and green to blue (best yield class observed).  Yield class classification was based on \ProductivityIndexText{} as given by \Cref{eq:NagelFunctionSolvedForProductivityIndex} (rounded to one decimal digit), using \Cref{tab:SchoberProductivityIndices} as reference.  Note the different yield class ranges in both plots.}
  \label{fig:SCAM1StandAgeBasalAreaObservationsPredictionsYieldClassClassification}
\end{figure}

\begin{figure}[h]
  \centering
  \includegraphics[width=1.0\textwidth]{../../Graphics/Thesis/SCAM1/SCAM1TopHeightBasalAreaObservationsPredictionsYieldClassClassification.pdf}
  \caption{Basal area (\(G\)) over top height (\(\TopHeightMath{}\)) of \Beech{} (top) and \Spruce{} (bottom).  Colored lines, dots, black lines, and colors have the same meaning as in \Cref{fig:SCAM1StandAgeBasalAreaObservationsPredictionsYieldClassClassification}, namely:  Colored lines represent predictions of model SCAM1.  Each dot represents one observation.  Black lines connect observations belonging to the same sample plot.  Color signifies the (fractional) yield class of the respective observation or prediction, ranging from red (worst yield class observed) over yellow and green to blue (best yield class observed).  Yield class classification was based on \ProductivityIndexText{} as given by \Cref{eq:NagelFunctionSolvedForProductivityIndex} (rounded to one decimal digit), using \Cref{tab:SchoberProductivityIndices} as reference.  Note the different yield class ranges in both plots.}
  \label{fig:SCAM1TopHeightBasalAreaObservationsPredictionsYieldClassClassification}
\end{figure}

\clearpage{}

%%% Local Variables:
%%% mode: latex
%%% TeX-master: "MasArThesis.tex"
%%% End:
