\subsection{GAM2}

\Cref{fig:GAM2EffectStandAgeVariable} shows the estimated effect of the stand age variable smooth function over the stand age variable for both species in model GAM2.  General characteristics of the curves are similar to those in \Cref{fig:GAM1EffectStandAgeVariable}:  for \Beech{}, the curve is markedly concave, reaching its maximum at a approximately \(\StandAgeVariableMath{} = \SI{29}{\meter}\), which corresponds to an age of circa \SI{85}{\year};  for \Spruce{}, the curve exhibits a steep increase up to roughly \(\StandAgeVariableMath{} = \SI{15}{\meter}\), which is equivalent to an age of about \SI{30}{\year}, after which the curve’s slope is reduced but remains positive.  As in \Cref{fig:GAM1EffectStandAgeVariable}, in \Beech{} the estimated effect covers a narrower range of values than in \Spruce{} and does not show signs of asymptotic behavior in either species.  The \edf{} of the smooth functions were \num{3.7} and \num{5.32} for \Beech{} and \Spruce{}, respectively.

\Cref{fig:GAM2EffectProductivityIndexVariable} depicts the estimated partial effect of the \ProductivityIndexVariableText{} over the \ProductivityIndexVariableText{} for both species in model GAM2.  The effect is monotone increasing in both species.  However, the slope of \Beech{} line is steeper than that of \Spruce{}.

\Cref{fig:GAM2QQPlot} shows the quantile-quantile plot for model GAM2 for both species.  In \Beech{}, several residuals lie outside the \SI{90}{\percent} reference band, scattered across the whole range of theoretical quantiles.  In \Spruce{}, a few  low quantile residuals lie outside the reference band.

\Cref{fig:GAM2FittedValuesResiduals} shows response residuals over fitted values of model GAM2 for both species.   In \Beech{}, residual variance appears to be constant over the range of fitted values depicted, whereas in \Spruce{}, residual variance appears to be higher for fitted values around \num{50} and lower for other values.

\Cref{fig:GAM2StandAgeBasalAreaObservationsPredictionsYieldClassClassification} depicts basal area over stand age, both observed values as well as predictions of model GAM2.  General model behavior is different between species:  in \Beech{}, prediction curves are concave, reaching their maximum around a stand age of \SI{80}{\year};  in \Spruce{}, prediction curves are increasing throughout the depicted range of stand ages, with a steep increase from \SIrange{0}{30}{\year} and a slower increase thereafter.  In both species, curves are clearly stratified depending on yield class, with curve order following yield class order at all times, i.e., a yield class shows higher basal area predictions than the next best yield class.

\Cref{fig:GAM2TopHeightBasalAreaObservationsPredictionsYieldClassClassification} shows basal area over top height, both observed values as well as predictions of model GAM2.  As in \Cref{fig:GAM2StandAgeBasalAreaObservationsPredictionsYieldClassClassification}, the \Beech{} plot exhibits clear stratification of prediction curves depending on yield class.  For ages up to \SI{20}{\year}, curve order is the inverse of yield class order. Between \SIrange{20}{33}{\year}, curve order switches, so that for ages above \SI{30}{\year}, curve order is the same as yield class order.  In \Spruce{}, curve stratification is visible as well, but curve order is highly erratic, with yield class -2 performing worst and yield class 4 performing best for almost the entire range of stand age depicted.

\begin{figure}[h]
  \centering
  \includegraphics[width=1.0\textwidth]{../../Graphics/Thesis/GAM2/GAM2EffectStandAgeVariable.pdf}
  \caption{Estimated effect of the stand age variable smooth function (\(s(\StandAgeVariableMath{}, \ldots)\)) over stand age variable (\(\StandAgeVariableMath{}\)) in model GAM2 for \Beech{} (top) and \Spruce{} (bottom).   Solid lines, dashed lines, vertical bars, and numbers in the y-axis titles have the same meaning as in \Cref{fig:GAM1EffectStandAgeVariable}, namely:  Solid lines mark estimates.  Dashed lines mark confidence bands of 2 standard errors width.  Vertical bars mark observed values.  Numbers in the y-axis titles are the effective degrees of freedom of the smooth function.}
  \label{fig:GAM2EffectStandAgeVariable}
\end{figure}

\begin{figure}[h]
  \centering
  \includegraphics[width=1.0\textwidth]{../../Graphics/Thesis/GAM2/GAM2EffectProductivityIndexVariable.pdf}
  \caption{Estimated partial effect of the parametric productivity index variable term (Partial for \(\ProductivityIndexVariableMath{}\)) over \ProductivityIndexVariableText{} (\(\ProductivityIndexVariableMath{}\)) in model GAM2 for \Beech{} (top) and \Spruce{} (bottom).   Solid lines, dashed lines, and vertical bars have the same meaning as in \Cref{fig:GAM1EffectStandAgeVariable}, namely:  Solid lines mark estimates.  Dashed lines mark confidence bands of 2 standard errors width.  Vertical bars mark observed values.  Note the different scaling of the x-axis in both plots.}
  \label{fig:GAM2EffectProductivityIndexVariable}
\end{figure}

\begin{figure}[h]
  \centering
  \includegraphics[width=1.0\textwidth]{../../Graphics/Thesis/GAM2/GAM2QQPlot.pdf}
  \caption{Quantile-quantile plot of observed values minus predicted values (Response residuals) over corresponding prediction quantiles (Theoretical quantiles) of model GAM2 for \Beech{} (top) and \Spruce{} (bottom).  Black lines, black dots, and gray shaded areas have the same meaning as in \Cref{fig:GAM1QQPlot}, namely:  Black lines are 1:1 reference lines.  Black dots are residuals.  Gray shaded areas mark reference bands between the \num{0.05} and \num{0.95} quantiles of predictions (\SI{90}{\percent} level).  Theoretical quantiles and reference bands are based on repeated predictions (\(N = \num{e4}\)) \parencite{Augustin2012}.  Note the different axis scaling in both plots.}
  \label{fig:GAM2QQPlot}
\end{figure}

\begin{figure}[h]
  \centering
  \includegraphics[width=1.0\textwidth]{../../Graphics/Thesis/GAM2/GAM2FittedValuesResiduals.pdf}
  \caption{Observed values minus fitted values (Response residuals) over fitted values of model GAM2 for \Beech{} (top) and \Spruce{} (bottom).}
  \label{fig:GAM2FittedValuesResiduals}
\end{figure}

\begin{figure}[h]
  \centering
  \includegraphics[width=1.0\textwidth]{../../Graphics/Thesis/GAM2/GAM2StandAgeBasalAreaObservationsPredictionsYieldClassClassification.pdf}
  \caption{Basal area (\(G\)) over stand age of \Beech{} (top) and \Spruce{} (bottom).  Colored lines represent predictions of model GAM2.  Dots, black lines, and colors have the same meaning as in \Cref{fig:StandAgeTopHeightYieldClassClassification}, namely:  Each dot represents one observation.  Black lines connect observations belonging to the same sample plot.  Color signifies the (fractional) yield class of the respective observation or prediction, ranging from red (worst yield class observed) over yellow and green to blue (best yield class observed). Yield class classification was based on \textcite{Schober1995} (moderate thinning).  Note the different yield class ranges in both plots.}
  \label{fig:GAM2StandAgeBasalAreaObservationsPredictionsYieldClassClassification}
\end{figure}

\begin{figure}[h]
  \centering
  \includegraphics[width=1.0\textwidth]{../../Graphics/Thesis/GAM2/GAM2TopHeightBasalAreaObservationsPredictionsYieldClassClassification.pdf}
  \caption{Basal area (\(G\)) over top height (\(\TopHeight{}\)) of \Beech{} (top) and \Spruce{} (bottom).  Colored lines, dots, black lines, and colors have the same meaning as in \Cref{fig:GAM2StandAgeBasalAreaObservationsPredictionsYieldClassClassification}, namely:  Colored lines represent predictions of model GAM2.  Each dot represents one observation.  Black lines connect observations belonging to the same sample plot.  Color signifies the (fractional) yield class of the respective observation or prediction, ranging from red (worst yield class observed) over yellow and green to blue (best yield class observed). Yield class classification was based on \textcite{Schober1995} (moderate thinning).  Note the different yield class ranges in both plots.}
  \label{fig:GAM2TopHeightBasalAreaObservationsPredictionsYieldClassClassification}
\end{figure}

\clearpage{}

%%% Local Variables:
%%% mode: latex
%%% TeX-master: "MasArThesis.tex"
%%% End:
