\subsection{Data sets}

The data set size for \Beech{} is reported in \Cref{tab:ObservationsCountPerEdvidBeech}.  The data set comprises 18 sample plots, with a total of 63 observations and a mean of \num{3.5} observations per sample plot.  The data set size for \Spruce{} is reported in \Cref{tab:ObservationsCountPerEdvidSpruce}.  The data set comprises 28 sample plots, with a total of 100 observations and a mean of \num{3.6} observations per sample plot.  In both data sets, the number of observations per sample plot ranges from \numrange{2}{8}.

\newpage{}  %% TESTING (this is meant only to prevent "misplaced \cr" errors due to the following longtable stretching across page breaks)
\begin{singlespace}
  {\tabulinesep=2mm
    \begin{longtabu}{l L S[table-figures-decimal = 0]}
      \caption{Number of observations per sample plot, total number of sample plots, and total number of observations in the \Beech{} data set. \label{tab:ObservationsCountPerEdvidBeech}} \\
      \toprule
      & Sample plot ID & {Number of observations} \\
      \midrule
      \endhead
      \bottomrule
      \endlastfoot
      & 00521004 & 4 \\
      & 04221005 & 5 \\
      & 08021003 & 2 \\
      & 58321003 & 8 \\
      & 8942102A & 4 \\
      & 8942102B & 2 \\
      & 89521002 & 5 \\
      & 89621002 & 4 \\
      & 89721006 & 2 \\
      & 90421001 & 2 \\
      & 99321000 & 4 \\
      & A1321300 & 2 \\
      & A8121011 & 2 \\
      & H1021001 & 2 \\
      & J5121001 & 4 \\
      & J5121005 & 5 \\
      & J5121007 & 4 \\
      & Z72NAT01 & 2 \\
      Total & 18 & 63 \\
      % Mean & & 3.5 \\
      % Median & & 4 \\
    \end{longtabu}
  }
\end{singlespace}

\newpage{}  %% TESTING (this is meant only to prevent "misplaced \cr" errors due to the following longtable stretching across page breaks)
\begin{singlespace}
  {\tabulinesep=2mm
    \begin{longtabu}{l L S[table-figures-decimal = 0]}
      \caption{Number of observations per sample plot, total number of sample plots, and total number of observations in the \Spruce{} data set. \label{tab:ObservationsCountPerEdvidSpruce}} \\
      \toprule
      & Sample plot ID & {Number of observations}  \\
      \midrule
      \endhead
      \bottomrule
      \endlastfoot
      & 05451102 & 5 \\
      & 06451102 & 5 \\
      & 07151102 & 8 \\
      & 07551103 & 7 \\
      & 07551105 & 3 \\
      & 4665111A & 2 \\
      & 4665112B & 2 \\
      & 4665113B & 2 \\
      & 4665114B & 4 \\
      & 4675112A & 2 \\
      & 4675113A & 3 \\
      & 4675113B & 3 \\
      & 47451104 & 5 \\
      & 55751102 & 3 \\
      & 87021515 & 2 \\
      & 87021517 & 2 \\
      & 87021522 & 2 \\
      & J6351121 & 2 \\
      & J6351141 & 3 \\
      & S1051103 & 3 \\
      & S1751101 & 3 \\
      & S1851101 & 4 \\
      & S1951101 & 3 \\
      & S2051102 & 3 \\
      & S2151101 & 2 \\
      & S2251101 & 5 \\
      & S2451102 & 4 \\
      & S2651104 & 8 \\
      Total & 28 & 100 \\
      % Mean & & 3.6 \\
      % Media & & 3 \\
    \end{longtabu}
  }
\end{singlespace}

The geographical location and the altitude above sea level of sample plots are reported in \Cref{fig:LocationsSamplePlots,fig:SpeciesAltitudeOfSamplePlots}, respectively.  The dots in the plots of both figures do not add up to the total number of sample plots of the respective species because some sample plots were part of the same trial and therefore shared the same geographical location and altitude.  In the \Beech{} data set, altitude above sea level spans \SI{525}{\meter}, ranging from \SIrange{40}{565}{\meter}, with a mean of \SI{313.2}{\meter}.  In the case of \Spruce{} it spans \SI{730}{\meter}, ranging from \SIrange{20}{750}{\meter}, with a mean of \SI{410.5}{\meter}.

\begin{figure}[h]
  \centering
  \includegraphics[width=1.0\textwidth]{../../Graphics/Thesis/LocationsSamplePlots.pdf}
  \caption{Geographical location of the sample plots of \Beech{} (left) and \Spruce{} (right) in Germany.}
  \label{fig:LocationsSamplePlots}
\end{figure}

\begin{figure}[h]
  \centering
  \includegraphics[width=0.5\textwidth]{../../Graphics/Thesis/SpeciesAltitudeOfSamplePlots.pdf}
  \caption{Altitude above sea level of sample plots.}
  \label{fig:SpeciesAltitudeOfSamplePlots}
\end{figure}

In \Cref{fig:StandAgeTopHeightYieldClassClassification,fig:StandAgeProductivityIndexYieldClassClassification,fig:StandAgeBasalAreaYieldClassClassification}, dot color signifies the (fractional) yield class of the respective observation.  Classification of observations into yield classes required multiple steps, which were undertaken separately for each species.  First, for all observations the \ProductivityIndexText{} (\(\ProductivityIndexMath{}\)) in \si{\meter} was computed using the equation
\begin{equation}
  \label{eq:NagelFunctionSolvedForProductivityIndex}
  % \ProductivityIndexMath{} = \frac{\TopHeightMath{}(x) + 49.872 - 7.3309 \cdot \ln(x) - 0.77338 \cdot \ln(x)^2 }{0.52684 + 0.10542 \cdot \ln(x)}
  \ProductivityIndexMath{} = \frac{\TopHeightMath{}(x) - \beta_0 - \beta_1 \cdot \ln(x) - \beta_2 \cdot \ln(x)^2 }{\beta_3 + \beta_4 \cdot \ln(x)},
\end{equation}
where \(\ProductivityIndexMath{}\) is top height (\(h_{100}\)) in \si{\meter} at age \SI{100}{\year}, \(x\) is stand age, and \(\TopHeightMath{}(x)\) is top height in \si{\meter} at age \(x\) and \(\beta_0, \ldots, \beta_4\) are species-specific coefficients \parencite{Nagel1999}.  The coefficients were taken from \textcite{Nagel1999} and are reported in \Cref{tab:NagelFunctionCoefficients}.  Next, a sequence of \(\ProductivityIndexMath{}\)-values was generated, ranging from yield class \num{4} over yield classes \numrange{3}{-1} to yield class \num{-2}, using an increment of \SI{0.1}{\meter}, thus also covering fractional yield classes.  Values of whole yield classes are reported in \Cref{tab:SchoberProductivityIndices}.  Values for yield classes \numrange{4}{1} were taken from \textcite{Schober1995} (moderate thinning), while values for yield classes \numrange{0}{-2} were linearly interpolated from those for yield classes \numlist{2;1}.   The sequence was then restricted to the range between the best yield class needed to include the lowest observed \(\ProductivityIndexMath{}\) and the worst yield class needed to include the highest observed \(\ProductivityIndexMath{}\).  A color palette matching this restricted sequence was then generated, ranging from red (worst yield class observed) over yellow and green to blue (best yield class observed).  Each observed \(\ProductivityIndexMath{}\) was then mapped to the color representing the corresponding \(\ProductivityIndexMath{}\)-value.

\begin{singlespace}
  {\tabulinesep=2mm
    \begin{longtabu}{L S[table-figures-decimal = 3] S[table-figures-decimal = 4] S[table-figures-decimal = 5] S[table-figures-decimal = 5] S[table-figures-decimal = 5]}
      \caption{Species-specific coefficients of \Cref{eq:NagelFunctionSolvedForProductivityIndex} as reported in \textcite{Nagel1999}.  \label{tab:NagelFunctionCoefficients}} \\
      \toprule
      Species & {\(\beta_0\)} & {\(\beta_1\)} & {\(\beta_2\)} & {\(\beta_3\)} & {\(\beta_4\)} \\
      \midrule
      \endhead
      \bottomrule
      \endlastfoot
      Beech & -75.659 & 23.192 & -1.468 & 0 & 0.2152 \\
      Spruce & -49.872 & 7.3309 & 0.77338 & 0.52684 & 0.10542 \\
    \end{longtabu}
  }
\end{singlespace}

\begin{singlespace}
  {\tabulinesep=2mm
    \begin{longtabu}{L S S S S S S S}
      \caption{Absolute productivity index of stand (\(\ProductivityIndexMath{}\)) of different yield classes (YC) for \Beech{} and \Spruce{}.  Values for yield classes \numrange{4}{1} were taken from \parencite{Schober1995} (moderate thinning).  Values for yield classes \numrange{0}{-2} were linearly interpolated from those of yield classes \numlist{2;1}.  \label{tab:SchoberProductivityIndices}} \\
      \toprule
      & \multicolumn{7}{c}{{\(\ProductivityIndexMath{}\) [\si{\meter}]}} \\
      \cline{2-8}
      Species & {YC 4} & {YC 3} & {YC 2} & {YC 1} & {YC 0} & {YC -1} & {YC -2} \\
      \midrule
      \endhead
      \bottomrule
      \endlastfoot
      Beech & 20.7 & 24.7 & 28.6 & 32.4 & 36.2 & 40 & 43.8 \\
      Spruce & 23.5 & 27.2 & 31.2 & 35.1 & 39 & 42.9 & 46.8 \\
    \end{longtabu}
  }
\end{singlespace}

As can be seen from \Cref{fig:StandAgeTopHeightYieldClassClassification}, several differences between the \Beech{} data set and the \Spruce{} data set exist.  In the former, stands are notably older and cover a wider range of ages, with stand age ranging from \SIrange{35}{153.6}{\year}, spanning \SI{118.6}{\year} with mean over all observations of \SI{76.3}{\year} and a mean over all sample plots of \SI{77.6}{\year}.  In the latter, age ranges from \SIrange{15}{113}{\year}, spanning \SI{98}{\year} with a mean over all observations of \SI{53.8}{\year} and a mean over all sample plots of \SI{53.9}{\year}.  Consequently, stands in the \Beech{} data set have a higher top height, with \(\TopHeightMath{}\) ranging from \SIrange{16.3}{39.2}{\meter} (difference: \SI{22.9}{\meter}), but \Spruce{} stands cover a slightly wider range of top heights, with \(\TopHeightMath{}\) ranging from \SIrange{9.1}{33.3}{\meter} (difference: \SI{24.2}{\meter}).   In the case of \Beech{}, the maximal top height of all observations was reached by a sample plot of intermediate yield class (\(\approx{} 1\)) compared to the other sample plots in the data set.   In the case of \Spruce{}, the maximal top height of all observations was reached by a sample plot of bad yield class (\(\approx{} 2\)) compared to the other sample plots in the data set.  Apart from one observation of \Beech{}, top height increases with stand age.

% \newpage{}
\begin{figure}[t]
  \includegraphics[width=1\textwidth]{../../Graphics/Thesis/StandAgeTopHeightYieldClassClassification.pdf}
  \caption{Observed relationship between stand age and top height (\(\TopHeightMath{}\)) for \Beech{} (top) and \Spruce{} (bottom).  Each dot represents one observation.  Black lines connect observations belonging to the same sample plot.  Dot color signifies the (fractional) yield class of the respective observation, ranging from red (worst yield class observed) over yellow and green to blue (best yield class observed).  Yield class classification was based on \textcite{Schober1995} (moderate thinning).  Note the different yield class ranges in both plots.}
  \label{fig:StandAgeTopHeightYieldClassClassification}
\end{figure}

\Cref{fig:StandAgeProductivityIndexYieldClassClassification} shows the observed development of \ProductivityIndexText{} \(\ProductivityIndexMath{}\) over stand age for \Beech{} (top) and \Spruce{} (bottom).  For all sample plots in both data sets, the \ProductivityIndexText{} changed at least once during stand development.  However, for several sample plots the direction of this change itself differs during stand development and there does not seem to be a general direction to which changes in \ProductivityIndexText{} adhered.  The \Beech{} data set covers a narrower range of productivity indices and consequently yield classes than the \Spruce{} one, with \(\ProductivityIndexMath{}\) ranging from \SIrange{23.7}{38.3}{\meter} (difference: \SI{14.6}{\meter}) and yield class ranging from \numrange{4}{-1} in the former, whereas for \Spruce{} \(\ProductivityIndexMath{}\) ranges from \SIrange{24.1}{45.2}{\meter} (difference: \SI{21.1}{\meter}) and yield class from \numrange{4}{-2}.

% \newpage{}
\begin{figure}[t]
  \includegraphics[width=1\textwidth]{../../Graphics/Thesis/StandAgeProductivityIndexYieldClassClassification.pdf}
  \caption{Observed relationship between stand age and \ProductivityIndexText{} (\(\ProductivityIndexMath{}\)) for \Beech{} (top) and \Spruce{} (bottom).  Dashed lines mark the \(\ProductivityIndexMath{}\) of yield classes.  Dots, black lines, and dot color have the same meaning as in \Cref{fig:StandAgeTopHeightYieldClassClassification}, namely:  Each dot represents one observation.  Black lines connect observations belonging to the same sample plot.  Dot color signifies the (fractional) yield class of the respective observation, ranging from red (worst yield class observed) over yellow and green to blue (best yield class observed). Yield class classification was based on \textcite{Schober1995} (moderate thinning).  Note the different yield class ranges in both plots.}
  \label{fig:StandAgeProductivityIndexYieldClassClassification}
\end{figure}

\Cref{fig:StandAgeBasalAreaYieldClassClassification} depicts the observed development of basal area \(G\) over stand age for \Beech{} (top) and \Spruce{} (bottom).
The \Beech{} data set has a higher minimal and a lower maximal basal area and covers a narrower range of basal areas compared to the \Spruce{} one, with \(G\) ranging from \SIrange{12.6}{51.3}{\square\meter} (difference: \SI{38.7}{\square\meter}) in the former, and from \SIrange{10.8}{79.8}{\square\meter} (difference: \SI{69}{\square\meter}) in the latter.

% \newpage{}
\begin{figure}[t]
  \includegraphics[width=1\textwidth]{../../Graphics/Thesis/StandAgeBasalAreaYieldClassClassification.pdf}
  \caption{Observed relationship between stand age and basal area (\(G\)) for \Beech{} (top) and \Spruce{} (bottom).  Dots, black lines, and dot color have the same meaning as in \Cref{fig:StandAgeTopHeightYieldClassClassification}, namely:  Each dot represents one observation.  Black lines connect observations belonging to the same sample plot.  Dot color signifies the (fractional) yield class of the respective observation, ranging from red (worst yield class observed) over yellow and green to blue (best yield class observed). Yield class classification was based on \textcite{Schober1995} (moderate thinning).  Note the different yield class ranges in both plots.}
  \label{fig:StandAgeBasalAreaYieldClassClassification}
\end{figure}

\Cref{fig:TopHeightBasalAreaYieldClassClassification} depicts the observed development of basal area over top height for \Beech{} (top) and \Spruce{} (bottom).  In all sample plots of both species, basal area increases as top height increases.

% \newpage{}
\begin{figure}[t]
  \includegraphics[width=1\textwidth]{../../Graphics/Thesis/TopHeightBasalAreaYieldClassClassification.pdf}
  \caption{Observed relationship between top height (\(\TopHeightMath{}\)) and basal area (\(G\)) for \Beech{} (top) and \Spruce{} (bottom).  Dots, black lines, and dot color have the same meaning as in \Cref{fig:StandAgeTopHeightYieldClassClassification}, namely:  Each dot represents one observation.  Black lines connect observations belonging to the same sample plot.  Dot color signifies the (fractional) yield class of the respective observation, ranging from red (worst yield class observed) over yellow and green to blue (best yield class observed). Yield class classification was based on \textcite{Schober1995} (moderate thinning).  Note the different yield class ranges in both plots.}
  \label{fig:TopHeightBasalAreaYieldClassClassification}
\end{figure}


One chief goal of this study was to ensure that the fitted models were capable of separating the effects of stand age and of \ProductivityIndexText{} on predicted basal area.  However, this requires that corresponding predictor variables are already available for model training.  Top height (\(h_{100}\)) was considered an unsuited predictor variable for this, since past experience had shown that using this dimension as the predictor variable does not allow identification of a separate productivity index-effect in the model, at least when data set size is rather limited as is the case in the present study.  Therefore, 2 new variables were calculated:  a stand age variable and a \ProductivityIndexVariableText{}, each of which was calculated in such a way as to exclude the effect of the other. The stand age variable was calculated using the equation
\begin{equation}
  \label{eq:StandAgeVariable}
  \StandAgeVariableMath{} = \beta_0 + \beta_1 \cdot \ln(x) + \beta_2 \cdot \ln(x)^2 + \ProductivityIndexYieldClassIMath{} \cdot \bigl(\beta_3 + \beta_4 \cdot \ln(x)\bigr),
\end{equation}
where \(\StandAgeVariableMath{}\) is the top height in \si{\meter} at age \(x\) if the stand were yield class 1, \(\ProductivityIndexYieldClassIMath{}\) is the species-specific top height (\(h_{100}\)) at age \SI{100}{\year} of a stand of yield class 1 and all other terms have the same meaning as in \Cref{eq:NagelFunctionSolvedForProductivityIndex}, namely: \(x\) is stand age and \(\beta_0, \ldots, \beta_4\) are species-specific coefficients which are reported in \Cref{tab:NagelFunctionCoefficients} \parencite{Nagel1999}.  The value for \(\ProductivityIndexYieldClassIMath{}\) was taken from \textcite{Schober1995} (moderate thinning) and was \SI{32.4}{\meter} for \Beech{} and \SI{35.1}{\meter} for \Spruce{}.  By setting \(\ProductivityIndexYieldClassIMath{}\) to a species-specific constant, it was possible to exclude any productivity index-effects from the stand age variable.  The independence of the stand age variable from productivity index-effects is made apparent by \Cref{fig:StandAgeStandAgeVariableYieldClassClassification}, which depicts the observed relationship between stand age and stand age variable:  for each species, all observations follow the same curve, regardless of yield class.  Notable differences with respect to the stand age variable exist between the data sets of both species:  for \Beech{} it ranges from \SIrange{13}{39}{\meter}, spanning \SI{26}{\meter} with a mean of \SI{25.6}{\meter}, while for \Spruce{} it ranges from \SIrange{4.2}{38.1}{\meter}, spanning \SI{33.9}{\meter} with a mean of \SI{23.8}{\meter}.  These range differences are not in correspondence with the differences in stand age between species:  compared to \Spruce{}, the \Beech{} data set covers a wider range of stand age but a narrower range of stand age variable;  the opposite is true for the \Spruce{} data set compared to \Beech{}.

\begin{figure}[t]
  \includegraphics[width=1\textwidth]{../../Graphics/Thesis/StandAgeStandAgeVariableYieldClassClassification.pdf}
  \caption{Observed relationship between stand age and stand age variable (\(\StandAgeVariableMath{}\)) for \Beech{} (top) and \Spruce{} (bottom).  Black lines depict the function given by \Cref{eq:StandAgeVariable}, which was used for calculating the observed values.  Dots and dot color have the same meaning as in \Cref{fig:StandAgeTopHeightYieldClassClassification}, namely:  Each dot represents one observation.  Dot color signifies the (fractional) yield class of the respective observation, ranging from red (worst yield class observed) over yellow and green to blue (best yield class observed).  Yield class classification was based on \textcite{Schober1995} (moderate thinning).  Note the different yield class ranges in both plots.}
  \label{fig:StandAgeStandAgeVariableYieldClassClassification}
\end{figure}

The \ProductivityIndexVariableText{} was calculated in 2 steps.  First, the \ProductivityIndexText{} was calculated using \Cref{eq:NagelFunctionSolvedForProductivityIndex}.  Subsequently, the \ProductivityIndexText{} of yield class 1 was subtracted from this value:
\begin{equation}
  \label{eq:ProductivityIndexVariable}
  \ProductivityIndexVariableMath{} = \ProductivityIndexMath{} - \ProductivityIndexYieldClassIMath{},
\end{equation}
where \(\ProductivityIndexVariableMath{}\) is the \ProductivityIndexVariableText{}, \(\ProductivityIndexMath{}\) has the same meaning as in \Cref{eq:NagelFunctionSolvedForProductivityIndex}, and \(\ProductivityIndexYieldClassIMath{}\) has the same meaning as in \Cref{eq:StandAgeVariable}, namely: it is the species-specific top height (\(h_{100}\)) at age \SI{100}{\year} of a stand of yield class 1 (\SI{32.4}{\meter} for \Beech{} and \SI{35.1}{\meter} for \Spruce{} according to \textcite{Schober1995} (moderate thinning)).  \(\ProductivityIndexVariableMath{}\) is a measure of the performance of a stand relative to a reference stand (here: a stand of yield class 1 according to \textcite{Schober1995}): high values signify stands which outperform the reference stand, whereas low values indicate low performance stands.  Since both \(\ProductivityIndexMath{}\) and \(\ProductivityIndexYieldClassIMath{}\) refer to a specific stand age (here: 100 \si{\year}) rather than a variable one, \(\ProductivityIndexVariableMath{}\) should only be influenced by stand productivity and largely be free of any stand age-effects.  \Cref{fig:StandAgeProductivityIndexVariableYieldClassClassification} depicts the observed relationship between stand age and \ProductivityIndexVariableText{}.  Despite the attempt to exclude stand age-effects from it, the \ProductivityIndexVariableText{} does exhibit an evident negative correlation with stand age, in that older stands tend to have lower \(\ProductivityIndexVariableMath{}\) values. The correlation coefficient between stand age and \ProductivityIndexVariableText{} was \num{-0.665} for \Beech{} and \num{-0.737} for \Spruce{}.  However, no individual stand shows a clear worsening of \ProductivityIndexVariableText{} over time.  Therefore, it seems plausible that the correlation is due to the fact that older stands (which were established several decades ago) were not subject to the rather recent phenomenon of nitrogen fertilization through atmospheric deposition \parencite{Kenk1988}, rather than being due to a true dependence of \ProductivityIndexVariableText{} on stand age.  The differences in observed yield classes between both species are also encountered in the \ProductivityIndexVariableText{}:  for \Beech{} it ranges from \SIrange{-6.7}{8.6}{\meter}, spanning \SI{15.3}{\meter} with mean of \SI{1.1}{\meter}, while for \Spruce{} it ranges from \SIrange{-11}{10.1}{\meter}, spanning \SI{21.1}{\meter} with a mean of \SI{-0.2}{\meter}.

\begin{figure}[t]
  \includegraphics[width=1\textwidth]{../../Graphics/Thesis/StandAgeProductivityIndexVariableYieldClassClassification.pdf}
  \caption{Observed relationship between stand age and \ProductivityIndexVariableText{} (\(\ProductivityIndexVariableMath{}\)) for \Beech{} (top) and \Spruce{} (bottom).  Dots, black lines, and dot color have the same meaning as in \Cref{fig:StandAgeTopHeightYieldClassClassification}, namely:  Each dot represents one observation.  Black lines connect observations belonging to the same sample plot.  Dot color signifies the (fractional) yield class of the respective observation, ranging from red (worst yield class observed) over yellow and green to blue (best yield class observed).  Yield class classification was based on \textcite{Schober1995} (moderate thinning).  Note the different yield class ranges in both plots.}
  \label{fig:StandAgeProductivityIndexVariableYieldClassClassification}
\end{figure}

% \newpage{}  %% TESTING (this is meant only to prevent "misplaced \cr" errors due to the following longtable stretching across page breaks)
% \begin{singlespace}
  % {\tabulinesep=2mm
    % \begin{longtabu}{L S S}
      % \caption{Summary statistics of the altitude above sea level of the sample plots. \label{tab:AltitudeSummaries}} \\
      % \toprule
      % Statistic & {Beech} & {Spruce} \\
      % \midrule
      % \endhead
      % \bottomrule
      % \endlastfoot
      % Minimum & 40 & 20 \\
      % Mean & 315.6 & 404 \\
      % Median & 363.5 & 445 \\
      % Maximum & 565 & 750 \\
    % \end{longtabu}
  % }
% \end{singlespace}

\clearpage{}

%%% Local Variables:
%%% mode: latex
%%% TeX-master: "MasArThesis.tex"
%%% End:
