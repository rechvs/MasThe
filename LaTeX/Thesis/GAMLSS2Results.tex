\subsection{GAMLSS2}

\Cref{fig:GAMLSS2EffectStandAgeVariable} shows the estimated effect of the stand age variable smooth function over stand age variable in model GAMLSS2 of both species.  The smooth function used unconstrained P-splines as its basis.  For both \Beech{} and \Spruce{}, the curve’s shape is very similar to the corresponding plot in \Cref{fig:GAMLSS1EffectStandAgeVariable}.  For \Beech{}, the effect has a concave shape, reaching its maximum at a stand age variable value of approximately \SI{31}{\meter}, which corresponds to an age of about \SI{93}{\year}.  For \Spruce{}, initially the curve has a rather steep positive slope, between stand age variable values of \SIrange{5}{12.5}{\meter} (which corresponds to an age range of \SIrange{16}{25}{\year}).  Between \SIrange{12.5}{20}{\meter} (\SIrange{25}{40}{\year}), the slope gradually decreases, while remaining positive. At roughly \SI{32.5}{\meter} (\SI{83}{\year}) the slope increases again for the remainder of the plot.

\Cref{fig:GAMLSS2EffectProductivityIndexVariable} shows the estimated linear effect of the \ProductivityIndexVariableText{} smooth function over \ProductivityIndexVariableText{} in model GAMLSS2 of both species.  In both species the effect is a monotone increasing line with its root at approximately \SI{0}{\meter}.

\Cref{fig:GAMLSS2QQPlot} shows the quantile-quantile plot of model GAMLSS2 for both species.  In \Beech{}, 2 residuals deviate outside the \SI{90}{\percent} reference band, whereas in \Spruce{} 7 residuals do.  In either species, deviations from the reference line occur in the lower quantiles only, while the residuals of higher quantiles adhere fairly closely to it.

\Cref{fig:GAMLSS2FittedValuesResiduals} depicts the normalized quantile residuals over fitted values of model GAMLSS2 for both species.  In both species, residual variance appears to be constant over the range of fitted values depicted.

\Cref{fig:GAMLSS2StandAgeBasalAreaObservationsPredictionsYieldClassClassification} shows basal area over stand age, both observations as well as predictions of model GAMLSS2 for both species.  In both species, the general shape of the prediction curves is similar to that in \Cref{fig:GAMLSS1StandAgeBasalAreaObservationsPredictionsYieldClassClassification}:  for \Beech{}, curves have a concave shape, each reaching its maximum at approximately \SI{90}{\year};  for \Spruce{}, curves have a rather steep slope between ages \SIrange{15}{25}{\year}, after which the slope gradually decreases between ages \SIrange{25}{40}{\year}, while remaining positive, until at approximately age \SI{90}{\year} slope gradually increases again.  In \Beech{}, predicted basal area initially decreases between ages \SIrange{3}{5}{\year}, before it increases until reaching its maximum for either yield class.  As in \Cref{fig:GAMLSS1StandAgeBasalAreaObservationsPredictionsYieldClassClassification}, prediction curves are stratified depending on yield class.  A major difference to \Cref{fig:GAMLSS1StandAgeBasalAreaObservationsPredictionsYieldClassClassification}, however, is that for both species, curve order at all times is the same as yield class order, meaning that any yield class shows higher predicted basal area than the next worse yield class.

\Cref{fig:GAMLSS2TopHeightBasalAreaObservationsPredictionsYieldClassClassification} depicts basal are over top height, both observations as well as predictions of model GAMLSS2 for both species.  In \Beech{}, prediction curves have generally a concave shape, each reaching its maximum at a different top height.  For top heights up to approximately \SI{22.5}{\meter}, curve order is the inverse of yield class order, with yield class \num{3} performing best and yield class \num{-2} performing worst.  Between top heights of \SIrange{22.5}{36}{\meter}, curve order switches, so that for top heights above \SI{36}{\meter}, curve order is the same as yield class order, with yield class \num{-2} performing best and yield class \num{3} performing worst.  In \Spruce{}, prediction curves have a slanted wave-like form.  Curve order generally is the inverse of yield class order, with predictions of adjacent yield classes becoming almost identical between top heights of \SIrange{13}{43}{\meter} and yield class \num{4} always outperforming yield class \num{-2} for a given top height.

\begin{figure}[h]
  \centering
  \includegraphics[width=1.0\textwidth]{../../Graphics/Thesis/GAMLSS2/GAMLSS2EffectStandAgeVariable.pdf}
  \caption{Estimated effect of the stand age variable smooth function (\(ps(\StandAgeVariableMath{})\)) over stand age variable (\(\StandAgeVariableMath{}\)) in model GAMLSS2 for \Beech{} (top) and \Spruce{} (bottom).  Solid lines, dashed lines, and vertical bars have the same meaning as in \Cref{fig:GAM1EffectStandAgeVariable}, namely:  Solid lines mark estimates.  Dashed lines mark confidence bands of 2 standard errors width.  Vertical bars mark observed values.  Note the different axis scaling in both plots.}
  \label{fig:GAMLSS2EffectStandAgeVariable}
\end{figure}

\begin{figure}[h]
  \centering
  \includegraphics[width=1.0\textwidth]{../../Graphics/Thesis/GAMLSS2/GAMLSS2EffectProductivityIndexVariable.pdf}
  \caption{Estimated effect of the \ProductivityIndexVariableText{} smooth function (\(ps(\ProductivityIndexVariableMath{})\)) over \ProductivityIndexVariableText{} (\(\ProductivityIndexVariableMath{}\)) in model GAMLSS2 for \Beech{} (top) and \Spruce{} (bottom).  Solid lines, dashed lines, and vertical bars have the same meaning as in \Cref{fig:GAMLSS2EffectStandAgeVariable}, namely:  Solid lines mark estimates.  Dashed lines mark confidence bands of 2 standard errors width.  Vertical bars mark observed values.  Note the different axis scaling in both plots.}
  \label{fig:GAMLSS2EffectProductivityIndexVariable}
\end{figure}

\begin{figure}[h]
  \centering
  \includegraphics[width=1.0\textwidth]{../../Graphics/Thesis/GAMLSS2/GAMLSS2QQPlot.pdf}
  \caption{Quantile-quantile plot of the normalized quantile residuals of model GAMLSS2 for \Beech{} (top) and \Spruce{} (bottom). Solid lines, black dots, and dashed lines have the same meaning as in \Cref{fig:GAMLSS1QQPlot}, namely:  Solid lines are reference lines.  Black dots represent residuals.  Dashed lines mark reference bands between the \num{0.05} and \num{0.95} quantiles of predictions (\SI{90}{\percent} level).  For a definition of normalized quantile residuals see \textcite{Dunn1996}.  Reference bands are based on the standard errors of the order statistics of an independent random sample from the standard normal distribution \parencite{Fox2016}.  Note the different axis scaling in both plots.}
  \label{fig:GAMLSS2QQPlot}
\end{figure}

\begin{figure}[h]
  \centering
  \includegraphics[width=1.0\textwidth]{../../Graphics/Thesis/GAMLSS2/GAMLSS2FittedValuesResiduals.pdf}
  \caption{Normalized quantile residuals (Quantile residuals) over fitted values of model GAMLSS2 for \Beech{} (top) and \Spruce{} (bottom).  For a definition of normalized quantile residuals see \textcite{Dunn1996}.}
  \label{fig:GAMLSS2FittedValuesResiduals}
\end{figure}

\begin{figure}[h]
  \centering
  \includegraphics[width=1.0\textwidth]{../../Graphics/Thesis/GAMLSS2/GAMLSS2StandAgeBasalAreaObservationsPredictionsYieldClassClassification.pdf}
  \caption{Basal area (\(G\)) over stand age of \Beech{} (top) and \Spruce{} (bottom).  Colored lines represent predictions of model GAMLSS2.  Dots, black lines, and colors have the same meaning as in \Cref{fig:StandAgeTopHeightYieldClassClassification}, namely:  Each dot represents one observation.  Black lines connect observations belonging to the same sample plot.  Color signifies the (fractional) yield class of the respective observation or prediction, ranging from red (worst yield class observed) over yellow and green to blue (best yield class observed). Yield class classification was based on \textcite{Schober1995} (moderate thinning).  Note the different yield class ranges in both plots.}
  \label{fig:GAMLSS2StandAgeBasalAreaObservationsPredictionsYieldClassClassification}
\end{figure}

\begin{figure}[h]
  \centering
  \includegraphics[width=1.0\textwidth]{../../Graphics/Thesis/GAMLSS2/GAMLSS2TopHeightBasalAreaObservationsPredictionsYieldClassClassification.pdf}
  \caption{Basal area (\(G\)) over top height (\(\TopHeightMath{}\)) of \Beech{} (top) and \Spruce{} (bottom).  Colored lines, dots, black lines, and colors have the same meaning as in \Cref{fig:GAMLSS2StandAgeBasalAreaObservationsPredictionsYieldClassClassification}, namely:  Colored lines represent predictions of model GAMLSS2.  Each dot represents one observation.  Black lines connect observations belonging to the same sample plot.  Color signifies the (fractional) yield class of the respective observation or prediction, ranging from red (worst yield class observed) over yellow and green to blue (best yield class observed). Yield class classification was based on \textcite{Schober1995} (moderate thinning).  Note the different yield class ranges in both plots.}
  \label{fig:GAMLSS2TopHeightBasalAreaObservationsPredictionsYieldClassClassification}
\end{figure}

\clearpage{}

%%% Local Variables:
%%% mode: latex
%%% TeX-master: "MasArThesis.tex"
%%% End:
