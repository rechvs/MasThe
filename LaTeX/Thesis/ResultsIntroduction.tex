Summary statistics for all models are reported in the Appendix in \Cref{tab:StatisticsGAM1Beech,tab:StatisticsGAM1Spruce,tab:StatisticsGAM2Beech,tab:StatisticsGAM2Spruce,tab:StatisticsSCAM1Beech,tab:StatisticsSCAM1Spruce,tab:StatisticsGAMLSS1Beech,tab:StatisticsGAMLSS1Spruce,tab:StatisticsGAMLSS2Beech,tab:StatisticsGAMLSS2Spruce,tab:StatisticsGAMLSS3Beech,tab:StatisticsGAMLSS3Spruce}.  All linear effects are significant at the \SI{5}{\percent} level.
The AIC of all \num{6} models for each species is reported in \Cref{tab:AICOverview}.  The lowest AIC for \Beech{} and \Spruce{} is achieved by models GAMLSS1 (\num{380.189}) and GAMLSS3 (\num{705.311}), respectively.

\begin{table}[H]
  {\tabulinesep=2mm
    \begin{longtabu}{L c p{3.5cm} c}  %% The "p" column allows manual adjustment of the horizontal spacing between the beech and spruce columns.
      \caption{Akaike Information Criterion (AIC) scores \parencite{Akaike1998} of all models for \Beech{} and \Spruce{}.
        Numbers in parentheses give the model rank, from lowest to highest AIC.
        \label{tab:AICOverview}} \\
      \toprule
      & \multicolumn{3}{c}{{AIC}} \\
      \cline{2-4}
      Model ID & {Beech} & & {Spruce} \\
      \midrule
      \endhead
      \bottomrule
      \endlastfoot
      GAM1 & 400.109 (3) & & 705.998 (2) \\
      GAM2 & 430.491 (5) & & 718.058 (6) \\
      SCAM1 & 442.729 (6) & & 717.144 (5) \\
      GAMLSS1 & 380.189 (1) & & 707.125 (3) \\
      GAMLSS2 & 399.319 (2) & & 710.280 (4) \\
      GAMLSS3 & 412.600 (4) & & 705.311 (1) \\
      \bottomrule
    \end{longtabu}}
\end{table}

In the following sections, the results of each model will be presented via the following plots:
\begin{itemize}
\item Estimated effect of the stand age variable term over stand age variable.
\item Estimated effect of the \ProductivityIndexVariableText{} term over \ProductivityIndexVariableText{}.
\item Quantile-quantile plot of model residuals.
\item Model residuals over fitted values.
\item Observed and predicted basal area over stand age.
\item Observed and predicted basal area over top height.
\end{itemize}

Data on which model predictions are based were generated in 2 steps. First, using an increment of one year, a sequence of age values was generated, ranging from \SIrange{2}{160}{\year}.  Then, corresponding sequences of top height values were calculated using the equation
\begin{equation}
  \label{eq:NagelFunctionSolvedForTopHeight}
  \TopHeightMath{}(x) = \beta_0 + \beta_1 \cdot \ln(x) + \beta_2 \cdot \ln(x)^2 + \ProductivityIndexMath{}_i \cdot (\beta_3 + \beta_4 \cdot \ln(x)),
\end{equation}
where
\(\ProductivityIndexMath{}_i\) is the absolute productivity index of stand for yield class \(i\) as reported in \Cref{tab:SchoberProductivityIndices},
with \(i\) ranging from worst to best observed yield class (\numrange{3}{-2} for \Beech{}, \numrange{4}{-2} for \Spruce{}),
and all other terms have the same meaning as in \Cref{eq:NagelFunctionSolvedForProductivityIndex}, namely:
\(\TopHeightMath{}(x)\) is top height in meter at age \(x\),
\(x\) is stand age,
and \(\beta_0, \ldots, \beta_4\) are species-specific coefficients as reported in \Cref{tab:NagelFunctionCoefficients} \parencite{Nagel1999}.
Based on these top height sequences, corresponding sequences of stand age variable and \ProductivityIndexVariableText{} were calcluated using \Cref{eq:StandAgeVariable,eq:ProductivityIndexVariable}, respectively.  These were then used as input for model predictions.

%%% Local Variables:
%%% mode: latex
%%% TeX-master: "MasArThesis.tex"
%%% End:
