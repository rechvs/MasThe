The results of each model are presented via the following plots:
\begin{itemize}
\item Estimated effect of the first predictor term over the corresponding variable.
\item Estimated effect of the second predictor term over the corresponding variable.
\item Q-Q-plot of sample quantiles over theoretical quantiles.
\item Link-residuals over link-fitted values. 
\item Observed and predicted basal area over stand age.
\item Observed and predicted basal area over top height.
\end{itemize}

Data on which model predictions are based were generated in 2 steps. First, a sequence of age values was generated, ranging from \SIrange{0}{160}{\year}.  Then, a corresponding sequence of top height values was calculated using equation
\begin{equation}
  \label{eq:NagelFunctionSolvedForTopHeight}
  \TopHeight{}(x) = \beta_0 + \beta_1 \cdot \ln(x) + \beta_2 \cdot \ln(x)^2 + \ProductivityIndexMath{}_i \cdot (\beta_3 + \beta_4 \cdot \ln(x)),
\end{equation}
where \(\TopHeight{}(x)\) is top height in \si{\meter} at age \(x\), \(x\) is stand age, \(\ProductivityIndexMath{}_i\) is the absolute productivity index of stand of yield class \(i\) as reported in \Cref{tab:SchoberProductivityIndices}, and \(\beta_0, \ldots, \beta_4\) are species-specific coefficients as reported in \Cref{tab:NagelFunctionCoefficients} \parencite{Nagel1999}.  Subsequently, corresponding sequences of stand age variable and \ProductivityIndexVariableText{} were calcluated using \Cref{eq:StandAgeVariable,eq:ProductivityIndexVariable}, respectively.  These were then used as input for model predictions.

Unless otherwise noted, the confidence bands of the estimated smooth function effects included non-zero values of the estimate for all smooth functions in all models for both species.  This suggests that both the stand age variable as well as the \ProductivityIndexVariableText{} are related to basal area in both species \parencite{Wood2001}.

%%% Local Variables:
%%% mode: latex
%%% TeX-master: "MasArThesis.tex"
%%% End:
