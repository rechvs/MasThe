\subsection{Comparison to other approaches}

While basal area modeling, both at individual tree level (e.g., \textcite{Andreassen2003,Hein2006,Jogiste2000,Monserud1996,Nystroem1997,Schroeder2002,Wimberly1996}) as well as stand level (e.g., \textcite{Castedo-Dorado2007,Chikumbo2001,Chikumbo1999,Eerikaeinen2003}), has been the subject of several studies, modeling of \emph{maximum} basal area has not gathered as much attention.  To the best knowledge of the author, only two approaches have been made (\textcite{Sterba1975,Sterba1987,Sterba1981} and \textcite{Woerdehoff2016}), none of which take differences in stand productivity into account.
The method presented by Sterba results from applying the ``competition-density rule'' developed by \textcite{Kira1953,Ando1968,Ando1968a,Tadaki1963} on mean diameter as suggested by \textcite{Goulding1972}, rather than on mean volume.  It allows prediction of maximum basal area using the equation
% \begin{equation}
  % \label{eq:SterbaNGmax}
  % \SterbaNGmaxMath{} = \frac{b_0}{a_0} \TopHeightMath{} ^ {b_1 - a_1},
% \end{equation}
\begin{equation}
  \label{eq:SterbaGmax}
  \SterbaGmaxMath{} = \frac{\pi}{16~a_0~b_0} \TopHeightMath{} ^ {-(a_1 + b_1)},
\end{equation}
% and
% \begin{equation}
  % \label{eq:SterbadgGmax}
  % \SterbadgGmaxMath{} = \frac{1}{2 b_o} \TopHeightMath{} ^ {-b_1},
% \end{equation}
% where \(\SterbaNGmaxMath{}\) is stand density at maximum basal area, \(a_0, a_1, b_0, b_1\) are coefficients to be estimated, \(\TopHeightMath{}\) is top height, \(\SterbaGmaxMath{}\) is maximum basal area, and \(\SterbadgGmaxMath{}\) is the diameter of the mean basal area tree.
where \(\SterbaGmaxMath{}\) is maximum basal area, \(a_0, a_1, b_0, b_1\) are coefficients to be estimated, and \(\TopHeightMath{}\) is top height.
In order to use \Cref{eq:SterbaGmax} for predicting maximum basal area, the equation needs to be fit to top height observations of maximum basal area stands in order to obtain valid estimates of \(a_0, a_1, b_0, \text{ and } b_1\).   \Cref{eq:SterbaGmax} uses only top height as an independent variable, thus offering no option to incorporate stand productivity as a predictor.
As a consequence, predictions obtained from \Cref{eq:SterbaGmax} are strictly only valid for stands which are of approximately the same yield class as the observations used for fitting it.  In order to obtain maximum basal area predictions for different yield classes, a new set of equation coefficients needs to be estimated, based on a suitable data set of top height and maximum basal area observations for the respective yield class.
Fitting of \Cref{eq:SterbaGmax} to the data sets used in the present study was not successful.  In order to nevertheless compare predictions of the most plausible models presented in this study (GAMLSS3 for \Beech{} and SCAM1 for \Spruce{}) with predictions of \Cref{eq:SterbaGmax}, coefficient estimates reported by \textcite{Doebbeler2004} (Region Nordwest) and \textcite{Woerdehoff2016} for \Beech{} and \Spruce{}, which are given in \Cref{tab:SterbaCoefficientsEstimatesDöbbeler,tab:SterbaCoefficientsEstimatesWördehoff}, respectively, were used .  The results are shown in \Cref{fig:TopHeightBasalAreaObservationsGAMLSS3SCAM1SterbaPredictionsYieldClassClassification}.  In the case of \Beech{}, \Cref{eq:SterbaGmax}, particularly when using the \textcite{Doebbeler2004} estimates, predicts notably higher basal areas for lower top heights than does model SCAM1 for yield classes \numrange{3}{-2}.  This relation then switches with increasing top height, so that for top heights above \SI{30}{\meter} and below \SI{50}{\meter} \Cref{eq:SterbaGmax} predicts lower basal areas than does model SCAM1 for yield class \num{-2}.  Unlike those of model SCAM1, however, predictions of \Cref{eq:SterbaGmax} continue to increase at a top height of \SI{60}{\meter}.  In the case of \Spruce{}, relations between predictions of \Cref{eq:SterbaGmax} and of model GAMLSS3 behave similarly to those described above.  For lower top heights, the former predicts higher basal areas than does the latter for yield classes \numrange{2}{-2}.  For higher top heights, \Cref{eq:SterbaGmax} predicts lower basal areas than does GAMLSS3 for yield classes \numrange{3}{-2} (\textcite{Woerdehoff2016} estimates) or for yield classes \numrange{0}{-2} (\textcite{Doebbeler2004} estimates).  \\
\textcite{Woerdehoff2016} used a GAMLSS (termed ``GAMLSSW'' in the following text) for modeling maximum basal area.  The model assumes basal area to follow Box-Cox-Cole-Green distribution, with identity function, natural logarithm function, and identiy function as the link function for \(\mu\), \(\sigma\), and \(\nu\), respectively.  Similar to \Cref{eq:SterbaGmax}, it only contains top height as a dependent variable.  Like GAMLSS1 and GAMLSS2, it uses P-splines as the function basis for the smooth function.  \Cref{fig:TopHeightBasalAreaObservationsGAMLSS3SCAM1WoerdehoffPredictionsYieldClassClassification} shows the predictions of GAMLSSW fitted to the data sets used in the present study in comparison to predictions of models GAMLSS3 for \Beech{} and SCAM1 for \Spruce{}.  GAMLSSW, due to its smooth function basis not being constrained, predicts increasing basal areas with increasing top height.  In the case of \Beech{}, the first third of the prediction curve somewhat resembles the curves of GAMLSS3, reaching a basal area minimax point of approximately \SI{36}{\square\meter\per\hectare} at a top height of roughly \SI{27}{\meter}.  The overall shape of the  prediction curve of GAMLSSW for \Spruce{} is similar to that for \Beech{}, albeit more compressed and having an inflection point, rather than a minimax point, at a basal area of around \SI{49}{\square\meter\per\hectare} and a top height of roughly \SI{23}{\meter}.  Up to a top height of \SI{35}{\meter} (\Beech{}) and \SI{28}{\meter} (\Spruce{}), GAMLSSW predictions are largely lower than those of GAMLSS3 for \Beech{} or SCAM1 for \Spruce{}.  For higher top heights, GAMLSSW predictions gradually exceeds all predictions for all depicted yield classes of the latter two models.  \\
While models SCAM1 and GAMLSS3 do suffer from certain inconsistencies discussed above, they nevertheless offer greater flexibility in terms of incorporating stand productivity as a predictor of maximum basal area than do \Cref{eq:SterbaGmax} or GAMLSSW.

\newpage{}  %% TESTING
\begin{singlespace}
    {\tabulinesep=2mm
    \begin{longtabu}{L S[table-format = 1.6e-1] S[table-format = 1.6] S[table-format = 1.6] S[table-format = -1.6]}  %% Note: there seems to be no way around manually setting the "table-format", since siunitx apparently has no mechanism for guessing the correct settings and the default settings (which are equivalent to "table-format = 3.2e0", see manual p. 44) are almost guaranteed to be inappropriate.
      \caption{Estimates of the coefficients of \Cref{eq:SterbaGmax} for \Beech{} as reported in \textcite{Doebbeler2004} (Region Nordwest) and \textcite{Woerdehoff2016}.
        \label{tab:SterbaCoefficientsEstimatesDöbbeler}} \\
      \toprule
      Source & {\(a_0\)} & {\(a_1\)} & {\(b_0\)} & {\(b_1\)} \\
      \midrule
      \endhead
      \bottomrule
      \endlastfoot
      \textcite{Doebbeler2004} & 1.0829e-7 & 1.5374 & 8.3652 & -1.7365 \\
      \textcite{Woerdehoff2016} & 2.616551e-7 & 1.368151 & 6.496417 & -1.731867 \\
      \end{longtabu}
    }
\end{singlespace}

\begin{singlespace}
    {\tabulinesep=2mm
    \begin{longtabu}{L S[table-format = 1.6e-1] S[table-format = 1.6] S[table-format = 1.6] S[table-format = -1.6]}  %% Note: there seems to be no way around manually setting the "table-format", since siunitx apparently has no mechanism for guessing the correct settings and the default settings (which are equivalent to "table-format = 3.2e0", see manual p. 44) are almost guaranteed to be inappropriate.
      \caption{Estimates of the coefficients of \Cref{eq:SterbaGmax} for \Spruce{} as reported in \textcite{Doebbeler2004} (Region Nordwest) and \textcite{Woerdehoff2016}.
        \label{tab:SterbaCoefficientsEstimatesWördehoff}} \\
      \toprule
      Source & {\(a_0\)} & {\(a_1\)} & {\(b_0\)} & {\(b_1\)} \\
      \midrule
      \endhead
      \bottomrule
      \endlastfoot
      \textcite{Doebbeler2004} & 1.28745e-6 & 0.7148 & 1.2842 & -1.1914 \\
      \textcite{Woerdehoff2016} & 4.913256e-6 & 0.4394706 & 0.3716977 & -0.9097641 \\
      \end{longtabu}
    }
\end{singlespace}

\begin{figure}[h]
  \centering
  \includegraphics[width=1.0\textwidth]{../../Graphics/Thesis/TopHeightBasalAreaObservationsGAMLSS3SCAM1SterbaPredictionsYieldClassClassification.pdf}
  \caption{Basal area (\(G\)) over top height \(\TopHeightMath{}\) of \Beech{} (top) and \Spruce{} (bottom). Depicted are observations as well as predictions of 3 different models per plot.  Colored lines represent predictions of model GAMLSS3 (top) or of model SCAM1 (bottom).  Dashed lines represent predictions of \Cref{eq:SterbaGmax}, using the coefficient estimates reported by \textcite{Woerdehoff2016} for the respective species.  Dotted lines represent predictions of \Cref{eq:SterbaGmax}, using the coefficient estimates reported by \textcite{Doebbeler2004} (Region Nordwest) for the respective species.  Dots, solid black lines, and colors have the same meaning as in \Cref{fig:GAMLSS3TopHeightBasalAreaObservationsPredictionsYieldClassClassification}, namely:  Each dot represents one observation.  Solid black lines connect observations belonging to the same sample plot.  Color signifies the (fractional) yield class of the respective observation or prediction, ranging from red (worst yield class observed) over yellow and green to blue (best yield class observed).  Yield class classification was based on \ProductivityIndexText{} as given by \Cref{eq:NagelFunctionSolvedForProductivityIndex} (rounded to one decimal digit), using \Cref{tab:SchoberProductivityIndices} as reference.  Note the different yield class ranges in both plots.}
  \label{fig:TopHeightBasalAreaObservationsGAMLSS3SCAM1SterbaPredictionsYieldClassClassification}
\end{figure}

\begin{figure}[h]
  \centering
  \includegraphics[width=1.0\textwidth]{../../Graphics/Thesis/TopHeightBasalAreaObservationsGAMLSS3SCAM1WoerdehoffPredictionsYieldClassClassification.pdf}
  \caption{Basal area (\(G\)) over top height \(\TopHeightMath{}\) of \Beech{} (top) and \Spruce{} (bottom). Depicted are observations as well as predictions of 2 different models per plot.  Colored lines represent predictions of model GAMLSS3 (top) or of model SCAM1 (bottom).  Dashed lines represent predictions of the GAMLSS reported by \textcite{Woerdehoff2016} fitted to the respective species’s data set.  Dots, solid black lines, and colors have the same meaning as in \Cref{fig:GAMLSS3TopHeightBasalAreaObservationsPredictionsYieldClassClassification}, namely:  Each dot represents one observation.  Solid black lines connect observations belonging to the same sample plot.  Color signifies the (fractional) yield class of the respective observation or prediction, ranging from red (worst yield class observed) over yellow and green to blue (best yield class observed).  Yield class classification was based on \ProductivityIndexText{} as given by \Cref{eq:NagelFunctionSolvedForProductivityIndex} (rounded to one decimal digit), using \Cref{tab:SchoberProductivityIndices} as reference.  Note the different yield class ranges in both plots.}
  \label{fig:TopHeightBasalAreaObservationsGAMLSS3SCAM1WoerdehoffPredictionsYieldClassClassification}
\end{figure}

% \clearpage{}  %% This is necessary, because otherwise the following "longtabu" environment produces a spurious table, which contains only the head but no cells.

\clearpage{}

%%% Local Variables:
%%% mode: latex
%%% TeX-master: "MasArThesis.tex"
%%% End:
