\subsection{Maximum basal area modeling}

The term “maximum basal area” refers to the site-specific highest possible basal area of living trees.  It plays a key role in forest management and planning, e.g., as a reference for determining thinning intensity \parencite{Assmann1961,Doebbeler2002} or stand density \parencite{Spellmann1999} or as an estimator of yield level \parencite{Franz1967}.  At the same time, experimental plots on which maximum basal area is recorded are few and far between and cover only a narrow selection of stand properties.  Thus, the need for extrapolating maximum basal area for stands for which not comparable experimental plot exists arises.  This study aims to adress this need.

%%% Local Variables:
%%% mode: latex
%%% TeX-master: "MasArThesis.tex"
%%% End:
