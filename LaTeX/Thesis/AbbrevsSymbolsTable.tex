\pagestyle{abbrevs_symbols}

{
  \renewcommand{\MakeUppercase}[1]{#1} % http://tex.stackexchange.com/questions/179966/fancyhdr-chaptermark-and-table-of-contents

  \newcommand{\mysectiontitle}{Table of abbreviations and symbols} % set section title as a custom macro to ease mutlitple usage (necessary when using a starred sectioning macro)
  
  \section*{\mysectiontitle{}}
  \label{sec:AbbrevsSymbolsTable}
  
  \markboth{\mysectiontitle{}}{} % http://tex.stackexchange.com/questions/89914/chapter-name-in-the-header-with-chapter
  \addcontentsline{toc}{section}{\mysectiontitle{}} % http://tex.stackexchange.com/questions/35433/creating-unnumbered-chapters-sections-plus-adding-them-to-the-toc-and-or-header
  
  % \begin{table}[H]
  %   \centering
  \begin{longtabu}{l L l}
    \toprule
    Symbol & Description & Unit \\
    \midrule
    \endfirsthead
    Symbol & Description & Unit \\
    \midrule
    \endhead
    \bottomrule
    \endlastfoot
    \(CV(\uplambda)\) & cross validation sum of squares for smoothing parameter \(\uplambda\) & depends on response variable \\
    \(D\) & average diameter (by basal area) & \si{\centi\meter} \\
    \(\mathbb{E}(x)\) & expectation of variable \(x\) & depends on \(x\) \\
    \(G\) & basal area & \si{\square\meter} \\
    GAM & generalized additive model & \\
    GCV & generalized cross validation & \\
    \(GCV(\uplambda)\) & generalized cross validation sum of squares for smoothing parameter \(\uplambda\) & depends on response variable \\
    GLM & generalized linear Model & \\
    \(\TopHeight\) & top height, height of dominant trees & \si{\meter} \\
    \(k\) & constant (varying with species) & \\
    LM & Linear Model & \\
    log & decadic logarithm & \\
    \(N\) & stand density & \si{\per\hectare} \\
    \(s\) & slope of the \logNlogDcurve{} \\
    \(SI\) & absolute site class & \si{\meter} \\
  \end{longtabu}
  % \end{table}
}

%% Reset table counter.
\setcounter{table}{0}

%%% Local Variables:
%%% mode: latex
%%% TeX-master: "MasArThesis.tex"
%%% End:
