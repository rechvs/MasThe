% NOTE: when inserting R code, escape special characters in the following way:
% - “_” = “\_”
% - “$” = “\$”
% - “^” = “\textasciicircum”
\section{Explanation of Data Frame Columns}

\subsection{SI.h100}
\textbf{Introduced with:} gmax\_1.5.RData

\textbf{Relevant R Code:}

\begin{rcode}
  bart\$SI.h100 <- \=(bart\$h100 + 49.87200 - 7.33090 * log(x = bart\$alt) \\
  \>- 0.77338 * ((log(x = bart\$alt))\textasciicircum 2.0)) \\
  \>/ (0.52684 + 0.10542 * log(x = bart\$alt))
\end{rcode}

\textbf{Explanation:} ``SI.h100'' holds the absolute stand index for a given stand (i.e., the value of h\textsubscript{100} at age \SI{100}{\years}) calculated with the function by Nagel (see email by Matthias Schmidt from 2017-04-27 12:06).

\subsection{h100.EKL.I}
\textbf{Introduced with:} gmax\_1.6.RData

\textbf{Relevant R Code:}

\begin{rcode}
  SI.h100.EKL.I <- 35.1 \\
  bart\$h100.EKL.I <- \=SI.h100.EKL.I * (0.52684 + 0.10542 \\
  \>* log(x = bart\$alt)) - 49.872 + 7.3309 \\
  \>* log(x = bart\$alt) + 0.77338 * (log(x = bart\$alt))\textasciicircum 2
\end{rcode}

\textbf{Explanation:} ``h100.EKL.I'' holds h\textsubscript{100} for a given age if the stand were EKL I.

\subsection{h100.diff.EKL.I}
\textbf{Introduced with:} gmax\_2.0.RData

\textbf{Relevant R Code:}

\begin{rcode}
  bart.clean\$h100.diff.EKL.I <- bart.clean\$h100.EKL.I - bart.clean\$h100
\end{rcode}

%%% Local Variables:
%%% mode: latex
%%% TeX-master: "MasAr_Notes"
%%% End:
