\section{Special remarks per edvid}

% NOTE: “Misplaced \cr” errors during compilation can be safely ignored.
\begin{singlespace}
  {\tabulinesep=2mm
    \begin{longtabu}{l p{0.8\linewidth}}
      \caption{Contents of \texttt{parz\$edvid} and \texttt{parz\$BESONDERHEITEN}.  Information on strength of thinning \textcolor{red}{highlighted}. \\
        Note:  ``strength of thinning'' and ``thinning intensity'' are NOT the same (cp. \textcite{Assmann1961}, p. 213 fn. 1). \label{tab:special_remarks}} \\
      \toprule
      \texttt{parz\$edvid} & \texttt{parz\$BESONDERHEITEN} \\
      \midrule
      \endfirsthead
      \texttt{parz\$edvid} & \texttt{parz\$BESONDERHEITEN} \\
      \midrule
      \endhead
      \bottomrule
      \endlastfoot
      05451102 & \\
      06451102 & 2000  intern aufgegeben \\
      07151102 & aufgeg. m. Schreiben v. 21.10.2009;1990 Kalkung \\
      07551103 & 1986 Kalkung; 2001 aufgegeben \\
      07551105 & 1977 aufgegeben:  keine ertragskundl. Aufnahme \\
      07551107 & 1986 Kalkung; 2012 aufgegeben \\
      11651100 & \\
      4665111A & 1977 Vollumbruch. unbehandelt. Erhebung Aststärkendurchmesser \\
      4665112B & 1977 Vollumbruch. \textcolor{red}{Auszeichnung nach Baumzahlleitkurve.} Erhebung Aststärkendurchmesser \\
      4665113B & 1977 Vollumbruch. \textcolor{red}{Auszeichnung nach Baumzahlleitkurve.} Erhebung Aststärkendurchmesser \\
      4665114B & 1977 Vollumbruch. \textcolor{red}{Auszeichnung nach Baumzahlleitkurve.} Erhebung Aststärkendurchmesser \\
      4675111A & unbehandelt. Erhebung Aststärkendurchmesser \\
      4675112A & unbehandelt. Erhebung Aststärkendurchmesser \\
      4675113A & unbehandelt. Erhebung Aststärkendurchmesser \\
      4675113B & ab 2009 \textcolor{red}{st. NDF} wegen WW (Kyrill) \\
      4675114A & unbehandelt. Erhebung Aststärkendurchmesser \\
      47451104 & Standort 09.3.2.4. z.T. 14.3.2.4. \textcolor{red}{Df. nach Baumzahlleitkurve} \\
      55751102 & gegattert \\
      56151100 & Standort 80\% 9.2.2.2. 20\% 15.2.2.2 \\
      61851101 & \textcolor{red}{keine Durchforstung.} da NP (Schutzzone 1). Düngungsangaben nachtragen !!! \\
      61851102 & \textcolor{red}{keine Durchforstung.} da NP (Schutzzone 1); ungedüngt \\
      87021515 & Prov.: Buche \\
      87021516 & Prov.: Buche \\
      87021517 & Prov.: Buche \\
      87021520 & Prov.: Buche \\
      87021521 & Prov.: Buche \\
      87021522 & Prov.: Buche \\
      A6251101 & Feinkartierung; Kompensationskalkung (3 t/ha) \\
      A6251104 & Feinkartierung; Kompensationskalkung (3 t/ha) \\
      J5851106 & Nullfläche \\
      J6351111 & \textcolor{red}{Df.art: starke Niederdurchforstung u. BZL} \\
      J6351121 & \textcolor{red}{Df.art: starke Niederdurchforstung u. BZL} \\
      J6351131 & \textcolor{red}{Df.art: starke Niederdurchforstung u. BZL} \\
      J6351141 & \textcolor{red}{Df.art: starke Niederdurchforstung u. BZL} \\
      J6551105 & \textcolor{red}{Durchforstung: starke Niederdurchfostung und Baumzahlleitkurve} \\
      J6551108 & \textcolor{red}{Durchforstung: starke Niederdurchfostung und Baumzahlleitkurve} \\
      S0651102 & 0 \\
      S1051103 & <NA> \\
      S1751101 & <NA> \\
      S1851101 & <NA> \\
      S1951101 & <NA> \\
      S2051102 & <NA> \\
      S2151101 & <NA> \\
      S2251101 & <NA> \\
      S2351103 & Schlußaufnahme 2013 \\
      S2451102 & <NA> \\
      S2551103 & <NA> \\
      S2651104 & <NA> \\
    \end{longtabu}
  }
\end{singlespace}

%%% Local Variables:
%%% mode: latex
%%% TeX-master: "MasAr_Notes"
%%% End:
